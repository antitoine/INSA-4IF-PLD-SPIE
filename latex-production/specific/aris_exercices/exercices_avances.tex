
\section{Généralités}

Les exercices débutants de la plateforme ARIS nous ont permis de déterminer la base des conditions d'un bon usage de la plateforme pour le Projet Longue Durée. Une ébauche de plan d'action a été mise en place afin de remédier aux éventuelles lacunes auxquelles les membres de l'hexanôme pourraient faire face, malgré la bonne réalisation des exercices.\newline
Afin de compléter ce plan d'action et de proposer une formation plus complète, présentant les fonctionnalités les plus avancées d'ARIS Architect, des exercices plus poussés ont été réalisés.

\section{Exercice 1}

Il peut être intéressant d'avoir accès à toutes les informations d'un groupe donné mais de vouloir voir seulement ses fonctions par exemple. Cet exercice nous a alors montré qu'il était très utile de pouvoir créer des modèles additionnels grâce au contenu élaboré depuis la base de données : la fonctionnalité de génération de modèle s'avère alors très utile afin de placer seulement des fonctions dans un modèle de type \og{}arbre de fonctions\fg{}. La découverte de l’outil \og{}Rechercher\fg{}, afin de rechercher un modèle ou type de modèle au sein d'une arborescence, est très utile. \newline
Aucune difficulté particulière n'a été rencontrée durant la réalisation de l'exercice. Le majeur problème serait plutôt de comprendre l'arbre de fonctions ainsi généré. Il serait intéressant de proposer une complément de formation sur cette fonctionnalité, qui peut s'avérer intéressante pour notre projet.

\section{Exercice 2}

Le but de cet exercice est d’utiliser un modèle déjà créé puis de le fusionner avec un modèle initialement généré. \'Etant donné que nous ne disposons que d’une seule base de données pour ce projet, cet exercice ne représente pas un grand intérêt dans notre cas (et je ne peux de toute façon pas le réaliser).

\section{Exercice 3}
    
Cet exercice présente un grand intérêt pour le projet futur car il nous permet de comparer deux versions d’un même modèle grâce à une fonctionnalité intégrée d’ARIS Architect. Le problème majeur de cette option est qu’elle me semble très lente (plus d’une minute pour réaliser le versionnage), ce qui peut être gênant dans le cas d’un projet. De plus, le bouton \og{}Comparer versions\fg{} était mystérieusement grisé. Il est devenu accessible après que j'ai fermé et rouvert la fenêtre... \newline
Autrement aucune difficulté n'a été rencontrée, les consignes de l’exercice étant plutôt limpides. Cette fonctionnalité fera l’objet d'un complément de formation car son utilité pour le projet n'est pas à démontrer.

\section{Exercice 4}

Cet exercice nous apprend comment générer des rapports, ce qui est indispensable pour la bonne conduite du projet. Cependant, les consignes sont très floues : je n’avais pas compris qu’il fallait ouvrir le modèle et pas simplement le sélectionner dans l’explorateur. \'Egalement, il n’était pas indiqué où le bouton \og{}Output model information\fg{} se trouvait, ce qui a entraîné une perte de temps. Enfin, le rapport généré est malheureusement loin d’être parfait d’un point de vue lisibilité. Il serait nécessaire de trouver les options permettant de l’afficher de façon correcte.

\section{Exercice 5}

Le fait que cet exercice nécessite des droits que je ne possède pas m’empêche de le réaliser. De plus, je ne pense pas que l’utilisation d’un filtre soit indispensable, car les différentes règles de modélisation permettant d’assurer une certaine cohérence pourront être mises en places entre nous, et je vérifierai régulièrement l’ensemble des productions de l’équipe quoi qu’il arrive. \\

\it{Les exercices suivants n’ont pas été réalisés car nous avons préféré privilégier l’avancement du projet principal.}

\section{Conclusion}

Ces différents exercices représentent un bon complément à proposer aux autres membres de mon hexanôme. En effet, la découverte de la fonctionnalité \og{}Rechercher\fg{} ainsi que la réalisation de rapports semblent indispensables au projet. De plus, la possibilité de versionner un modèle peut se révéler extrêmement utile si deux personnes travaillent dessus de façon asynchrone. On peut ainsi revenir en arrière si l’on se rend compte qu'une modification a été effectuée alors qu'elle n'avait pas lieu d'être. Encore une fois, il est dommage que les consignes ne soient pas toujours claires ou adaptées à notre configuration (une seule base de données, droits manquants…), ou encore que l'exercice avancé soit étroitement lié à l'exercice débutant.

