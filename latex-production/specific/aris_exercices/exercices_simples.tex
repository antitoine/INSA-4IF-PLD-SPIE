
\section{Généralités}

Afin de réaliser les différents exercices visant à redécouvrir la plateforme ARIS, l'outil ARIS Architect a été utilisé. Cependant, au premier abord, le mécanisme de lancement de l'application présente quelques difficultés. En effet, le lancement étant à effectuer via un navigateur internet, sur le réseau interne de l'INSA (l'accès par VPN étant impossible), des contraintes de localisation étaient fortement présentes. Dans un second temps, le lancement nécessitant l'exécution sur l'ordinateur client de code Java, quelques problèmes de compatibilité ont également été rencontrés. En effet, sur les machines du département informatique de l'INSA, les versions de Java n'étant pas forcément à jour, diverses alertes de sécurité peuvent être déclenchées au lancement de l'application ARIS Architect en fonction du navigateur web utilisé. \\

La seconde difficulté rencontrée sur cette plateforme concerne la connexion à la base de données et la visibilité des documents produits par les membres de l'hexanôme. En effet, tous les membres doivent se connecter sous la même langue, sous peine de ne pouvoir accéder aux documents produits en interne. Cette contrainte est également appliquée si un membre de l'équipe choisit malencontreusement \og{}Anglais USA\fg{} tandis qu'un autre choisit \og{}Anglais UK\fg{}. Cette subtilité devrait être mise en avant dès la connexion, afin d'éviter de perdre du temps à essayer de comprendre une potentielle erreur de configuration.

\section{Exercice 1}

Ayant déjà pratiqué de la modélisation ARIS durant notre troisième année à l'INSA, le premier exercice n'a pas présenté de difficulté particulière, mis à part la découverte (ou remémorisation) de l'interface du logiciel. Cependant, la représentation des liens entre les entités semble parfois ambiguë : par exemple \og{}is predecessor\fg{} ou \og{}process-oriented superior\fg{} sont représentés par un même figuré (flèche). Il faut aussi bien penser à préfixer les valeurs avec nos initiales afin de ne distinguer nos unités de celles du reste des membres de l'hexanôme. Ce sont les seuls points nous ayant fait perdre un peu de temps, les concepts devant être appliqués restant simples à ce stade.

\section{Exercice 2}

Cet exercice n'ayant pour objectif qu'une duplication de hiérarchie structurelle, l'intérêt pédagogique de celui-ci est assez limité, malgré un temps nécessaire non négligeable pour dupliquer l'ensemble des dossiers. De plus, étant donné que l'énoncé de l'exercice n'a pas été réalisé expressément pour les étudiants de l'INSA, les consignes sont floues : devait-on importer directement les dossiers d'UMG ? Ou les recréer nous-mêmes pas à pas ?

\section{Exercice 3}

L'exercice 3 a permis de redécouvrir comment un organigramme peut être modélisé en ARIS. Cependant, si quelques types de relations ont été évoqués, tous n'ont pas été abordés ni expliqués de nouveau. Redondant, cet exercice pourrait être moins long si certaines unités organisationnelles étaient écartées car non utilisées par la suite. Le point positif de l'exercice est la découverte des \og{}fragments\fg{}, concept dont nous n'avions jamais entendu parler. Il n'était cependant pas simple de trouver ensuite comment réimporter le fragment tout juste créé : un peu plus d’explications auraient été bonnes à prendre. \'Egalement, nous avons perdu du temps au début car nous ne nous souvenions plus que le copier-coller copiait l'objet en lui-même et pas simplement son figuré. Enfin, impossible de comprendre comment modifier un type de lien sans devoir supprimer l'entité qui lui est raccordée et devoir la recréer : ce serait quelque chose d'intéressant à savoir pour pouvoir modifier des liens sur des diagrammes plus complexes.

\section{Exercice 4}

Cet exercice a permis de découvrir le mécanisme d’attributs qui n'est pas très intuitif au premier regard de l'interface. Cependant, l'ajout de documents dans le modèle risque de le surcharger et de nuire à la lisibilité. L'exercice ne présentait finalement pas de difficulté particulière, certains membres de l'hexanôme ayant même trouvé que l'on était trop guidé et que l'on ne retenait donc pas très bien la procédure à suivre.

\section{Exercice 5}

Cet exercice nécessitant diverses autorisations d'accès et des documents qui ne sont pas accessibles actuellement, les différentes étapes ont donc été ignorées.

\section{Exercice 6}

La seule difficulté de cet exercice a été de bien penser à déplacer à la fois l'ensemble des diagrammes et les objets associés, qui n’apparaissent pas automatiquement dans l'arborescence (il faut cocher une case pour les faire apparaître). De plus, il est dommage de devoir déplacer tous les objets un par un et qu'il ne soit pas possible d'en sélectionner plusieurs à la fois : ce serait une fonctionnalité utile, si ce n'est nécessaire, à intégrer. Enfin, l'énoncé n'est pas très clair (ou bien l'arboresence incomplète), nous n'avons pas compris la troisième question.

\section{Exercice 7}

Il s'est avéré très utile de pouvoir rechercher une entité déjà créée ailleurs afin de pouvoir la placer dans un contexte différent ou la modifier. \\

\section{Exercice 8}

Cet exercice s'est avéré utile dans la mesure où il nous a permis de réaliser qu'ARIS Architect ne réalise pas de vérification concernant le respect des règles s'appliquant aux modèles, dans le cas de cet exercice un CPE. Cet outil devrait intégrer ce genre de fonctionnalités à partir du moment ou celle-ci permet d’éviter des erreurs d’inattention de la part de l'utilisateur. Il nous a également permis de nous remémorer les différentes règles régissant le CPE. \\

\it{Les exercices suivants n'ont pas été réalisés car nous avons préféré privilégier l'avancement du projet principal.} \\

\section{Conclusion}

Ces différents exercices se sont révélés être de bons rappels des bases d'ARIS et nous ont permis de nous familiariser avec l'interface d'ARIS Architect de façon relativement simple et sans que l’on reste bloqués trop longtemps sur un exercice, faute de trop peu d'informations par exemple. Il est tout de même dommage que certains exercices ne puissent pas être réalisés, comme le 5, car nous ne disposions pas de toutes les ressources nécessaires. Egalement, la charge de travail pour la réalisation de l'ensemble de ces travaux semble avoir été sous-estimée, ce pourquoi nous avons préféré ne pas réaliser tous les exercices mais plutôt en faire seulement quelques-uns de manière approfondie, en étudiant les fonctionnalités par nous-mêmes. En effet, certains exercices nous ont semblé légèrement superficiels et il aurait été intéressant d'en disposer d'un moins grand nombre mais qui nous poussent à aborder plus en profondeur les différentes options d'ARIS Architect, et plus globalement du langage ARIS.

