\section{Evaluation du PAQ par le Responsable Qualité}

Le Plan d’Assurance Qualité représente un atout important dans la démarche de mise en place de processus pour le déroulement du projet. Il fut très utile pour cerner les limites et les procédures d’élaboration des livrables sans se soucier du projet en lui même. Notre bon sens nous a poussé à faire un document complet et ayant comme objectif premier la qualité avant tout, sans penser aux conséquences comme le temps de réalisation ou l’attractivité des tâches. \\

Nous avons donc respecté ce procédé pour notre premier livrable, le dossier d’initialisation et nous nous sommes très vite rendu compte que c’était des tâches très lourdes et longues, dépassant largement les 10\% du temps prévu pour la qualité de ce livrable. Afin d'anticiper les problèmes que cela pouvait entraîner, nous avons revu le PAQ afin de réduire les tâches de relecture. Nous avons retiré les lectures croisées et laissé une relecture générale par le chef de projet ou le responsable qualité. Nous avons pu ainsi livrer dans les temps le dossier d’élaboration des besoins et les livrables suivants. La tâche de relecture se fait forcément au dernier moment et dépend donc du travail de toute l’équipe. Plus le contenu est terminé tard, plus le temps pour la qualité est diminué. \\

Enfin, la structuration des dossiers a permis de faire gagner un temps conséquent et la mise en place des plans pour chaque livrable a permis de focaliser plus rapidement les collaborateurs sur leurs tâches. De fait, se poser l’ensemble des questions pour réaliser le PAQ permet globalement une meilleure structuration du projet et une augmentation du coefficient qualité par rapport au temps passé.

\section{Bilan du Chef de Projet}

Ce bilan se divise en trois paragraphes traitant respectivement des points suivants : bilan du travail accompli, bilan humain et retour d’expérience concernant le projet en tant qu’étudiant. \\

En ce qui concerne le travail accompli, il est nécessaire de noter un certain nombre de points. Le premier étant que lors de la réalisation des deux premières phases, la compréhension du sujet était très superficielle et la qualité des livrables en sera sans doute la preuve. C’est lors de la seconde phase du projet que nous avons pris conscience du sens du travail que nous devions accomplir. Cela a eu un effet sur le moral de l’équipe et son travail.

Certaines procédures mises en place par le PAQ et notamment les relectures itératives ont été réduites le manque de temps en étant la cause principale. La phase d’initialisation s’est révélée complexe car le contexte du projet était très flou et nous devions définir et écrire ce que nous comptions faire. De plus le manque de connaissance du domaine métier auquel s’appliquait le projet n’a pas amélioré la situation. Si cette phase devait être effectuée aujourd’hui, il serait sans doute nécessaire de consacrer plus de temps à cette phase, deux séances plutôt qu’une, par exemple. \\

Le bilan humain est très positif, en effet, tous les membres de l’équipe se sont montrés responsables et ont assumé les tâches qui leur ont été confiées afin de permettre à l’ensemble du groupe de rendre les livrables dans les temps. Aucun accrochage ne s’est produit malgré la présence de forts caractères dans l’équipe. Cela a permis au groupe de travailler dans une ambiance saine, constructive et donc productive. Nous avons conscience que c’est un point exceptionnel vis-à-vis des autres hexanomes qui ont souvent été confrontés à des cas complexes à gérer. Cela n’a en rien pallié au besoin de cadrer les membres et de les guider tout au long du projet. La communication s’est révélée plutôt efficace. L’usage de moyens de communication modernes tels que le logiciel Slack mais également la séparation en sous groupes de plus petite taille lorsque les tâches le permettaient ont favorisé la communication. 

Aucun des membres de l’hexanôme n’était vraiment attiré par ce projet mais le sérieux et l’investissement personnel ont permis de palier à ce problème. Nous pensons que ce projet, même si l’intérêt que nous lui portions était faible voire nul, nous a apporté un certain nombre de notions et nous a permis d’acquérir de nouvelles compétences. Il nous a également rappelé un élément important de notre quotidien : nous ne faisons pas que des choses qui nous intéressent dans la vie en général. \\

En tant qu’étudiant, il me semble que ce projet nous apporte de nombreuses compétences mais manque d’intérêt pour plusieurs raison. SPIE est une entreprise bien trop grande, même en réduisant le périmètre aux activités de maintenance, pour un projet se réduisant à 9 séances. Nous n’avons pas le temps de nous approprier tous les éléments qui nous permettraient de mieux appréhender le travail à accomplir. Il serait donc intéressant de réduire la cible à une PME par exemple. Il est également assez frustrant de devoir concevoir une solution spécifique et donc une infrastructure physique sur laquelle s’appuie l’architecture applicative en ayant aucune idée de l’infrastructure existante de SPIE. Ce dernier point pourrait être corrigé en donnant une infrastructure fictive mais réaliste à l’échelle de SPIE. Pour finir, la dépendance à certains outils, bien que ceux-ci soient utiles et efficaces, entraîne de nombreuses complications au niveau de la réalisation.

\section{Bilans individuels et appréciations}

\subsection{Formateur ARIS : Lisa Courant}

\subsubsection{Retour d'expérience de Lisa}

La principale et première difficulté que j’ai rencontrée dans ce projet résidait en l’absence d’objectif vraiment clair et défini. Quel était notre rôle, qu’attendait-on de nous ? Quelles étaient les attentes de SPIE ? Tout ceci n’a été eclairci en partie que plus tard. À chaque séance, nous avons perdu beaucoup de temps à essayer de comprendre ce que l’on attendait de nous, quel était l’objectif de la séance et comment aborder les choses. \\

Je ne doute pas de l’utilité d’un tel projet, nous permettant d’aborder le SI d’une grande entreprise et d’en comprendre un peu mieux ses enjeux et son fonctionnement. Cependant, il aurait été de bon ton de définir de façon un peu plus claire les objectifs du projet et d’expliciter le métier de SPIE afin que l’on ne reste pas dans le flou aussi longtemps. \\

Ceci mis à part, j’aurais également apprécié de voir à quoi ressemblait un ERP et en avoir quelques exemples concrets avant que l’intégralité des cours n’ait été réalisée.

\subsubsection{Appréciation et note du Chef de Projet}

Lisa a fait preuve de beaucoup de sérieux concernant son rôle et a soutenu l’équipe lors des tâches de production. Elle a permis aux membre de travailler efficacement avec la plateforme ARIS. Aucun problème d’assiduité ou de comportement au sein du groupe n’est a déplorer.\\

\noindent\textsc{Note\footnote{La note est calculée en deux parties une note sur le comportement ($10/10$) et une sur le travail ($8/10$)} :} $18/20$

\subsection{Responsable Modélisation : Estelle Lepeigneux}

\subsubsection{Retour d'expérience d'Estelle}

Ce projet a été difficile à aborder pour tout le monde je pense. Nous avons eu beaucoup de mal a en saisir les enjeux, à comprendre le travail qui nous était demandé. Les premiers rapports ont été compliqués à aborder, j’avais un peu l’impression de devoir rédiger des pages et des pages de quelque chose auquel je ne comprenais rien. Finalement, après quelques séances et à force de discuter avec les membres de mon hexanôme et d’autres hexanômes, j’ai commencé à comprendre un peu mieux et à m’y mettre beaucoup plus sérieusement. J’ai ainsi pu travailler efficacement sur le projet, mais cette fois en comprenant ce sur quoi je travaillais. \\

Je pense que cette incompréhension est fortement liée au fait de ma non-motivation pour ce projet, mais j’ai malgré tout essayé de mettre cela de coté pour ne pas pénaliser le groupe. Cependant, on ne peut pas dire que le métier de SPIE était très explicite au premier abord, et cela nous a aussi posé pas mal de problèmes de compréhension. \\

Pour conclure, je pense que c’est intéressant de voir les différentes étapes d’un tel projet, même s’il est dommage de ne pas avoir de véritable contact avec les clients et un peu pompeux de se dire qualifié pour donner des conseils à une entreprise comme SPIE alors que c’est notre premier projet de la sorte et nous avons été dans le flou pendant un bon moment. Le projet est difficile, parce qu’on ne le comprends que tardivement, et l’impression de devoir écrire pour avoir un maximum de pages est omniprésente. C’est dommage car si on se concentrait sur la qualité et la compréhension, peut-être qu’on aurait saisi les enjeux plus tôt.

\subsubsection{Appréciation et note du Chef de Projet}

Estelle a fourni un soutien considérable lors de la réalisation des tâches de production et en particulier sur les tâches de modélisation. Aucun problème d’assiduité ou de comportement au sein du groupe n’est a déplorer.\\

\noindent\textsc{Note\footnote{La note est calculée en deux parties une note sur le comportement ($10/10$) et une sur le travail ($8/10$)} :} $18/20$

\subsection{Expert ERP : Hugues Verlin}

\subsubsection{Retour d'expérience de Hugues}

J'ai personnellement eu du mal a comprendre le sens de ce projet, surtout lorsque que l’on voit que tout est déjà possible de faire avec PeopleSoft. De même, tout ce qui concerne la gestion de projet est vraiment superflu, on le fait déjà dans les autres projets. \\

La documentation pour le projet est un peu trop éparpillée. De même, je n’ai pas trouvé très intéressant de se retrouver face à des documents comprenant des centaines de pages dont on a rien tirer. \\

Le dossier d’initialisation peut être supprimé car il n’apporte rien : on écrit que l’on va écrire des documents sans vraiment savoir de quoi on parle. Ensuite, il est ridicule de devoir recopier l’ensemble des informations qui sont déjà sur Moodle. \\

À la limite, il faudrait modifier l’exercice en donnant un faux cahier des charges de SPIE (ou d’une entreprise virtuelle avec de vrais problèmes, que SPIE n'a pas vraiment). \\

Enfin, il serait intéressant de voir à quoi ressemble un ERP (ainsi que les différents composants) avant de nous faire un cours dessus. \\

C’est globalement dommage car on passe moins de temps sur la partie la plus intéressante, l'\'Elaboration des Solutions.

\subsubsection{Appréciation et note du Chef de Projet}

Hugues a fourni un travail important et notamment lors des tâches de production nécessitant des recherches approfondies de ressources sur Internet. A supervisé et participé à la plupart des tâches ayant attrait à SAP. Aucun problème d’assiduité ou de comportement au sein du groupe n’est a déplorer.\\

\noindent\textsc{Note\footnote{La note est calculée en deux parties une note sur le comportement ($10/10$) et une sur le travail ($8/10$)} :} $18/20$

\subsection{Expert Métier : Pierre Jarsaillon}

\subsubsection{Retour d'expérience de Pierre}

Mon principal ressenti envers ce projet concerne dans un premier temps une certaine frustration au regard de sa définition et de sa présentation. En effet, les attentes du projet et les différentes tâches à effectuer paraissaient parfois floues. Si l’étude d’une entreprise existante devrait au contraire permettre de travailler sur des éléments plus concrets, la présentation initiale de SPIE Sud-Est semblait trop généraliste et trop longue pour, in fine, obtenir peu de renseignements utilisables vis-à-vis de notre projet. De même, les différentes indications réparties sur Moodle, notamment concernant les plans des livrables, étaient souvent en contradiction avec les éléments annoncés lors des réunions, qui étaient alors à privilégier, nous rendaient confus quant à la méthodologie à appliquer et renforçait notre sentiment d’obsolescence du projet. \\
 
Ces différentes difficultés, additionnées à la longue durée du projet, nous ont tout de même apporté de nouvelles connaissances, notamment dans la gestion de la motivation. Même en possédant le rôle d’« Expert métier », il m’a semblé nécessaire et normal de m’investir dans l’équipe de travail, tout en supportant les autres membres. \\
 
Malgré les failles de son introduction, l’étude d’une entreprise existante et de son métier d’activité, ou tout du moins d’un de ses métiers, m’est également apparu comme une initiation à l’étude multi-métier et à la compréhension des besoins clients, augmentant ainsi nos compétences d’adaptabilités. \\
 
Concernant les différentes phases du projet, ce fut celle d’expression des besoins qui m’est apparue comme étant la plus longue et complexe à mettre en \oe{}uvre. En effet, celle-ci a soulevé des incompréhensions du métier et des attentes de l’entreprise, baissant ainsi le moral de l’équipe. La présence des vacances de Noël se juxtaposant aux séances liées à cette phase a également eu pour effet d’interrompre notre immersion dans le projet, rendant la tâche d’autant plus difficile à réaliser. A contrario, les séances extrêmement rapprochées en fin de projet ont augmenté la charge de travail. Une nouvelle organisation des séances du projet serait à envisager pour une performance accrue des étudiants. \\

\subsubsection{Appréciation et note du Chef de Projet}

Élément moteur pour l’équipe, Pierre a fourni un travail remarquable tout au long du projet. Il est a l’initiative des croissants nous ayant restaurés à chaque séance.
Aucun problème d’assiduité ou de comportement au sein du groupe n’est a déplorer.\\

\noindent\textsc{Note\footnote{La note est calculée en deux parties une note sur le comportement ($10/10$) et une sur le travail ($8/10$)} :} $18/20$

\subsection{Responsable Qualité : Antoine Chabert}

\subsubsection{Retour d'expérience d'Antoine}

Ce projet a été un véritable défi pour moi. En effet, mon intérêt n’y était pas mais cela reste un travail obligatoire et il fallait trouver la motivation dans le peu de chose qui avait du sens à mon goût. Je n'y voyais aucune perspective d’avenir dans ce domaine, sachant que je suis plus destiné à une perspective de recherche purement technique. Cependant j’ai gardé l’esprit ouvert, du fait qu’il est possible dans notre avenir professionnel de rencontrer ce type de problème, j’ai joué le jeu et j’ai essayé d’y mettre de la volonté. \\

Du point de vue du projet en lui-même, la présentation des différentes tâches n’était pas toujours claire. Nous sommes globalement tous novices dans les rôles que nous avons eu et même si des réunions (notamment pour la qualité) ont été menées, il n’est pas toujours facile de trouver ce vers quoi il faut aller. La distance avec l’entreprise ainsi que la distance avec le réel étaient trop importantes. On a plus l’impression d’un travail de rédaction et de synthèse qu’une vraie recherche d’ingénieur dans un problème d’entreprise actuel. De vraies interactions avec le client permettraient de mieux définir les attentes. Chacun comprend à sa manière, c’est pourquoi un entretien de vive voix est souvent plus concret et productif qu’une recherche documentaire. \\

L’initiative des exercices sur ARIS et SAP était une bonne idée, mais des consignes plus claires et une diminution de la charge de travail sont à envisager. Tout le temps passé sur ces plates-formes pour former chaque membre de l’équipe n’a pas été consacré à l’élaboration des besoins. D’autant plus qu’une division du travail a été faite et par conséquent tout le monde n’a pas forcément travaillé avec ARIS ou SAP pour le projet en lui-même. \\

Enfin, le principe de notation qu’a dû subir chaque rôle (exercices ARIS/SAP, qualité, chef de projet) ne me semble pas judicieux. Dans la mesure où chaque membre a correctement fait son travail, ce type de procédé ne peut apporter que de la discorde dans l’équipe. Nous avons tous le même niveau d’expertise et de ce fait il est injustifiable de juger les autres. \\

Cependant, j’en tire quand même du positif. C’est un exercice formateur sur le contrôle de soi et sur le travail d’équipe. Le chef de projet a fait énormément d’effort pour nous pousser et pour aller de l’avant. Il a été rassurant sur le déroulement du projet et de prendre chaque problème un à un. De ce fait, l’attribution des rôles est très importante et a permis de donner du sens en chaque individu.

\subsubsection{Appréciation et note du Chef de Projet}

Soutien précieux au chef de projet, Antoine a réalisé conjointement avec ce dernier les tâches relevant de la qualité et s’est également investit dans les tâches de production.
Aucun problème d’assiduité ou de comportement au sein du groupe n’est a déplorer.\\

\noindent\textsc{Note\footnote{La note est calculée en deux parties une note sur le comportement ($10/10$) et une sur le travail ($8/10$)} :} $18/20$