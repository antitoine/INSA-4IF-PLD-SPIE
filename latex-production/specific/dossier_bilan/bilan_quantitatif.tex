\section{Synthèse des indicateurs}

Cette section présente le bilan qualitatif concernant le projet et l’analyse de ces derniers. Vous trouverez en annexe A le tableau de bord dans son état final. \\

\noindent Concernant les temps, les bilans par catégories sont présentés dans le tableau ci dessous :

\begin{table}[H]
    \caption{Tableau des indicateurs/mesures du projet}
    \begin{tabular}{p{5cm}|p{10cm}}
    ~ & \begin{tabular}{p{6cm}|p{4cm}}
        Mesure & Valeur (en heures) \\
        \end{tabular} \\ \hline
    Indicateurs projet & \begin{tabular}{p{6cm}|p{4cm}}
        Temps total estimé & 403,25 \\
        Temps total effectif & 303 \\ 
        Temps moyen/membre & 50,5 \\ 
        \end{tabular} \\ \hline
    Répartition par type de tâche & \begin{tabular}{p{6cm}|p{4cm}}
        Temps des tâches de production & 229,5 \\ 
        Temps de modélisation & 80 \\ 
        Temps des tâches de Qualité & ~25 \\ 
        Temps des tâches de Gestion de Projet & 48,5 \\ 
        \end{tabular} \\ \hline
    Répartition par phase & \begin{tabular}{p{6cm}|p{4cm}}
        Temps prise de connaissance estimé & 18,25 \\ 
        Temps prise de connaissance effectif & 27 \\ 
        Temps phase INIT estimé & 62 \\ 
        Temps phase INIT effectif & 59,75 \\ 
        Temps phase EB estimé & 149 \\ 
        Temps phase EB effectif & 93,75 \\ 
        Temps phase ES estimé & 115,5 \\ 
        Temps phase ES effectif & 54,5 \\ 
        Temps phase Eval. estimé & 40,5 \\ 
        Temps phase Eval. effectif & 25,5 \\
        Temps phase Bilan estimé & 18 \\ 
        Temps phase Bilan effectif & 16,5 \\ 
        \end{tabular} \\ \hline
    \end{tabular}
\end{table}

\section{Analyse des indicateurs}

Au regard du tableau précédent, nous constatons des écarts plus ou moins importants entre les temps envisagés et les temps effectifs, il est clair que certaines tâches ont été évaluées avec une marge importante. Les deux principales raisons ayant conduit à ces écarts sont les suivantes. La première est le manque d’expérience dans la réalisation d’un planning prévisionnel. La seconde est que les membres n’ont pas toujours été assidus sur la saisie de leurs temps respectifs et ont donc parfois saisi des temps approximatifs. \\

Concernant le projet dans son ensemble, nous constatons que la répartition des temps sur le projet par membre est à peu près équilibrée ce qui tend à confirmer le fait que toute l’équipe s’est impliquée du début à la fin du projet. \\

Dans l’ensemble, l’intégralité des échéances ont été respectées. Il n’est malheureusement pas possible de discerner le temps passé sur le projet en séance du temps passé sur le projet hors séance mais un rapide calcul nous donne pour 8 séances de 4h avec 6 membres dans l’équipe un total de 192h en séance. Il est clair que ce temps a été largement dépassé au final. Le conseil de ne pas passer plus de 2h sur le projet hors séance, si nous souhaitons atteindre un résultat acceptable concernant la qualité des livrables, était très optimiste. \\

Il est naturel que les tâches de production prennent le plus de temps. Il est difficile d’évaluer le temps passé sur les tâches de Gestion de Projet et de Qualité étant donné que celles-ci sont globalement traitées comme des tâches de fond et difficiles à nommer et à chiffrer car très variables. Il est tout de même possible de donner des ordres de grandeurs pour ces deux catégories. La rédaction de bilans hebdomadaire prend un temps considérable et n’apporte pas forcément une plus-value très élevée. En réalité ces bilans permettent de simuler des comptes rendus réguliers au client ou à la hiérarchie. \\

Les tâches ayant été réparties de manière équivalente au début du projet, il n’y a pas d’écart majeur à noter concernant ce point.  \\

L’indicateur du moral de l’équipe reste un outil intéressant et doit absolument être conservé. Renseigné avec honnêteté, il permet d’adapter les méthodes de management employées pour faire face aux situations complexes. L’analyse que nous pouvons faire de la courbe du moral sur ce projet est la suivante. Tout d’abord, il a été demandé aux membres de l’équipe d’évaluer leur moral sur une échelle entière allant de 0 à 5. La note spéciale 2,5 a été attribuée en cas d’absence. Nous  pouvons distinguer, sur cette courbe, trois temps majeurs. Le premier consiste en une lente décroissance sous la barre des 2,5 qui représente la chute de motivation dans l’équipe en fin d’année dernière. Elle s’explique en partie par la fatigue et les pressions extérieures s’étant exercées à cette période sur les membres. Ensuite nous observons un sursaut lors de la période des vacances. Enfin, une croissance importante en fin de projet. Cette hausse de moral s’explique de deux manières. La première étant l’approche de la fin du projet et la perspective heureuse de clore ce projet. La seconde le regain d’intérêt dans le projet par la compréhension tardive de l’intérêt du travail accompli lors de ce projet. L’importance de la transparence au sein de l’équipe a sans doute été un facteur clé du bon déroulement de ce projet, la feuille à renseigner étant accessible à tous et des commentaires pouvant être écrits pour justifier la note. \\

Globalement les indicateurs ont été d’une grande utilité et ont permis de conserver une certaine visibilité sur le projet. Des actions correctives ont été effectuées lors du constat d’un glissement s’étant produit durant la phase d’Expression des Besoin. \\