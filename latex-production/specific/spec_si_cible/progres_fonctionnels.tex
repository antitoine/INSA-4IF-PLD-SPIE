
\section{Suivi de l’activité et mise en place d’un tableau de bord}

Un autre axe intéressant à explorer peut être de mettre en place des indicateurs de performance. 
Ce type de tableau de bord permet à SPIE de pouvoir visualiser simplement l’évolution des performances de l’entreprise. De plus, si ce système est mis en place assez tôt, il permettra d’avoir un retour sur les impacts de la mise en place des autres propositions ici présentes. \\

Ce tableau de bord pourrait regrouper les informations suivantes : \\
    
Indicateurs de type organisationnel : il s’agit ici de relever le nombre d’interventions, le profil type des intervenants en fonction des situations, etc.\\
Indicateurs de type techniques : il s’agit ici d’indiquer la nature des travaux, le temps moyen passés sur un contrat, ou encore les moyens mobilisés\dots \\
Indicateurs économiques : on s’intéresse là à observer la marge, ou encore le rendements des contrats\dots \\

Toujours dans l’idée d’améliorer le suivi et afin de pouvoir estimer plus précisément les recettes de prestations, nous avons décidé de standardiser la phase d’achat des pièces de maintenance lors du lancement de prestation de services et de travaux. Ainsi, en estimant les pièces de rechange nécessaires après l’état des lieux du site, on peut vérifier leur disponibilité dans le stock, puis le réapprovisionner en amont de la réalisation des prestations.

\section{Évolution et mise en place de la mobilité}

Afin d’améliorer l’efficacité de ses processus et l’accès au reporting, il peut être intéressant pour SPIE de doter ses collaborateurs d’appareils mobiles ainsi que d’applications dédiées à ses besoins. En effet, la mise en place d’une application mobile pour les techniciens réalisant les différentes tâches de maintenance pourra favoriser la rédaction de retours d’expériences, directement à partir de leur terminal mobile. L’application, communiquant avec le système d’information de SPIE Sud-Est, permettra ainsi d’alimenter la base de connaissances de SPIE et permettant, à termes, d’exploiter les retours d’expérience afin d’améliorer la qualité des services fournis par SPIE. En effet, le technicien pouvant remplir sa fiche de retour d’expérience directement après son intervention, le risque d’omission est diminué, les événements et problématiques rencontrés étant encore récents. De plus, en cas de retour client, positif ou négatif, suite à l’intervention, le technicien peut adapter son retour d’expérience en fonction des impressions recueillies sur le terrain, augmentant davantage sa mobilité. \\

La mise en place d’une mobilité pour les techniciens a également pour avantage de leur permettre d’accéder aux différentes informations dont ils ont besoin sur le terrain, leur évitant ainsi d’accumuler des documents, au risque d’oublier, voir perdre, des informations importantes. Le gain de temps peut ainsi mettre mis à profit, permettant au technicien de se concentrer sur sa tâche, rehaussant de nouveau la qualité des services de maintenance et rejoignant un des objectifs principaux de SPIE Sud-Est, à savoir l’amélioration de la qualité des prestations. \\

De plus, l’application mobile mise en place pourra être modulaire, permettant ainsi de développer progressivement les modules correspondant aux besoins des différents acteurs de la société. Il faut également noter que l’utilisation d’une application mobile par les acteurs d’une société permet de renforcer l’image de celle-ci. En effet, cela véhicule une image interne et externe de modernité et de veille technologique.\\

La mise en place d’une application mobile au sein de SPIE Sud-Est, à destination dans un premier temps aux techniciens réalisant les interventions de maintenant, permettrait donc d’améliorer la communication avec le système d’information lors des interventions, mais également d’améliorer l’image de la société.\\

\section{Amélioration de la gestion des risques}

Actuellement, l’analyse des risques est effectuée ponctuellement, sans réel suivi sur le long terme. Il serait donc intéressant de mettre en oeuvre une gestion des risques plus poussée et plus régulière, qui interviendrait tout au long de la réalisation du contrat. Pour cela, il faudrait mettre en place un nouvel indicateur, l’indice de niveau de sécurité, qui interviendrait non seulement à des moments clés de la réalisation, mais également tout au long du processus. Ainsi, à chaque réalisation de prestation, une mise à jour des risques et, à fortiori, des indicateurs associés, sera à prévoir, après chaque revue périodique du contrat. De plus, un bilan des risques sera réalisé lors de la solde du contrat, afin de faire un point sur le déroulement de celui-ci. \\

Cette amélioration s’inscrit dans une optique de Business Intelligence, ou BI, en combinaison avec une base de connaissance. En effet, l’accumulation des données concernant la gestion des risques, grâce à une surveillance accrue, permettra de de tirer parti des différentes informations accumulées au fil des expériences. Tirer parti des éventuels problèmes rencontrés permettra ainsi de mieux prévoir les éventuels risques pouvant intervenir par la suite et, de fait, garantir une meilleure qualité de service. En effet, garder une trace des problèmes ayant survenus permet de proposer aux acteurs de la société les méthodes de résolution qui ont été mises en oeuvres précédemment. Cela permettra également de pouvoir mieux juger en amont les offres, lors des phases de négociation client. Ces dernières pourront ainsi être éventuellement jugées comme étant à risque potentiel et ainsi mettre en place une surveillance accrue du déroulement du contrat, si celui-ci est tout de même accepté. \\

La mise en place d’une gestion améliorée des risques permettrait donc à la fois d’augmenter la qualité des services proposés par SPIE Sud-Est, grâce à une augmentation de la satisfaction client, mais apporterait également une situation de confort aux acteurs de la société. En effet, la confrontation aux situations à risque étant partiellement prise en charge par cette nouvelle gestion des risques, les employés de SPIE Sud-Est sont alors soumis à moins de stress, améliorant ainsi leurs conditions de travail et, de fait, leur capacité de travail.
