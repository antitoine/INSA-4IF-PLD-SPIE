
\section{Favoriser la migration vers un nouveau système d’information uniformisé}

Comme mentionné au préalable dans la section Étude de l’existant, le système d’information actuel de SPIE est constitué de nombreuses applications différentes, s’intégrant plus ou moins difficilement à leur ERP actuel, PeopleSoft. Cependant, cette organisation induit des flux de données pouvant être hétérogènes car non normalisés, voir redondant. Cela provoque des difficultés de maintenance, avec des risques d’erreurs sur les données circulant sur le système d’information, tout en augmentant les coûts de maintenance. \\

L’étude réalisée ayant pour objectif de dégager de nouvelles solutions afin d’améliorer les activités de maintenance de SPIE, il est important d’avoir un système d’information modulaire, apte à intégrer de nouvelles modifications. Cependant, le passage d’un système d’information à un autre ne peut se faire sans planification précise. En effet, le système d’information actuel est composé d’un ERP, qui en constitue une grande partie, complexifiant davantage l’éventuelle migration vers un nouveau système. Sans planification détaillée et progressive, la migration peut provoquer des pertes financières, voir des pertes de données, qui peuvent se révéler plus problématiques à long terme. A l’heure actuelle, où le numérique est omniprésent dans les sociétés, les données numériques s’accumulent à grande vitesse et deviennent de vraies richesses pour les entreprises, qui peuvent ensuite les exploiter pour dégager des axes de développement. \\

Un remplacement abrupt de l’ERP PeopleSoft est également déconseillé vis-à-vis des utilisateurs, qui devront être formés à l’utilisation d’une nouvelle application, de grande complexité en termes de fonctionnalités. Un axe d’amélioration pouvant être envisagé, est donc une migration très progressive et partielle du système d’information actuel de SPIE, vers un nouveau système d’information normalisé. Cela permettrait de former au fur et à mesure les utilisateurs, tout en recueillant leurs impressions sur leur nouvelle plate-forme de travail, afin d’adapter si besoin les paramètres et personnalisations des nouvelles applications. Ainsi, une mise en place d’une architecture ERP 2-tiers, permettant de faire fonctionner les deux ERP simultanément (PeopleSoft et l’ERP retenu par la suite de l’étude), est à envisager, tout en permettant de budgétiser dans le temps la migration.

\section{Mise en place d’une base de connaissances}

La centralisation et la gestion des connaissances est primordiale pour une entreprise telle que SPIE, de même que l’est la capacité à pouvoir avoir des retours sur les missions qu’effectuent les techniciens lors des opérations de maintenance. \\

Les phases de reporting ne doivent plus être négligées et doivent faire l’objet de processus soigneusement étudiés afin d’être le plus utiles possibles, afin de faire ensuite gagner du temps à l’entreprise. \\

Ainsi, dans un objectif d’amélioration continue de la qualité de travail de ses collaborateurs, il peut être proposé à SPIE de mettre en place des fonctions de recueils des retours de l’ensemble des personnes travaillant à SPIE, mais également de récupérer un retour de la part de ses clients. \\

Il serait alors intéressant de coupler cette base avec une solution de Business Intelligence afin d’analyser aux mieux les retours. \\

Afin d’intégrer cette idée dans les processus de SPIE, nous allons ajouter les liens avec la base de connaissances, c’est-à-dire aux moments où nous allons la mettre à jour et l’utiliser dans le cadre de la Business Intelligence. On peut aussi mettre en place l’envoi d’un questionnaire de satisfaction client après chaque intervention pour avoir un retour du client sur les prestations de l’entreprise et ainsi de pouvoir évaluer la qualité de cette dernière.
