Cette étude, réalisée pour la société SPIE Sud-Est s'inscrit dans un projet de plus grande ampleur. Elle ne constitue en effet que la première étape d'un long processus de refonte des processus et du système d'information de l'entreprise SPIE Sud-Est. \\

Cette étude consistera en l'analyse des processus existant de SPIE ainsi que l'étude des solutions concurrente afin d'élaborer deux solutions pour refondre le système d'information de SPIE Sud-Est et par la même occasion éventuellement modifier ces processus internes de fonctionnement. \\

Ce document constitue l'étape de synthèse des éléments existant dans l'organisation SPIE Sud-Est, il a pour but de faire un bilan sur le système d'information et les processus tels qu'ils existent au moment où cette étude à lieu. Il décrit également l'ensemble des solutions techniques et informatiques déjà mises en place dans la société pour répondre aux attentes de cette dernière. \\

Ce document doit aussi permettre d'évaluer l'état de la société SPIE et d'identifier des axes d'amélioration. Il sera complété par la suite par deux documents qui sont décrit dans la suite.

Le premier document constitue l'évaluation des solutions mises en place par les concurrents de SPIE si ces inforrmations sont accessibles. Il présentera également les différents progiciels disponibles sur le marché et s'appuiera sur des processus formalisés afin d'évaluer l'adéquation de ces derniers avec les attentes de SPIE.

Le deuxième réalisera la synthèse des deux premiers en décrivant les pistes d'amélioration dégagées de l'étude de l'existant et du benchmarking précédemment réalisés. Il proposera alors une spécification de la cible à atteindre lors de la phase suivante qui consiste en l'élaboration de deux solutions. 