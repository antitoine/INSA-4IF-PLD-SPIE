\section{Présentation des processus métiers}

\begin{description}
\item \bf{Offre et revue d'offre} : Un appel d'offre est lancé, donc une opportunité de contrat de service est saisie. Après analyse, ils peuvent faire suite ou non à cet appel d'offre. \\

\item \bf{Négociation client} : Ce processus n'est pour l'instant pas décrit, on le verra dans un second temps avec les axes d'amélioration. \\

\item \bf{Commande et revue de commande} : Si l'offre a été validée avec le client, il faut réaliser la revue du dossier de commande et négocier avec le client. \\

\item \bf{Lancement des prestations de service et travaux} : On lance la commande après qu'elle ait été accepté. Il faut mobiliser les ressources nécessaires et créer des documents opérationnels pour aboutir à une situation initiale connue. \\

\item \bf{Réalisation de prestations de maintenance} : Il faut exécuter les prestations mises en place. Une revue périodique de contrat est effectuée pour ensuite se diriger vers les évolutions du contrat. \\

\item \bf{Réalisation travaux induits} : On réalise tous les travaux éventuellement induits par ce qui a été effectué précédemment. S'il n'y en a pas, on passe à la suite. \\

\item \bf{\'Evolution du contrat} : Le contrat peut éventuellement évoluer après analyse des risques. \\

\item \bf{Solde de l'affaire et du contrat} : Une fois que tout a été effectué, on va clore le contrat et archiver tous les documents.
\end{description}

\section{Détail des différents processus}

\subsection{Offre et revue d'offre}

À la suite d'un appel d'offres, une opportunité de contrat de service est saisie. Il faut également réaliser une opportunité d'avenant, faisant sortir l'entreprise du cadre de maintenance corrective pour aller vers un cadre de maintenance évolutive. Le directeur opérationnel et le responsable d'activité maintenance décideront de la saisie de l'opportunité, tandis que le directeur opérationnel et le commercial donneront leur avis. S'ils décident de ne pas la saisir, le pilote de l'offre indiquera au client qu'ils ne souhaitent pas poursuivre. À l'inverse, si l'opportunité est saisie, il faut tout d'abord collecter des données (à la responsabilité du pilote de l'offre), puis les analyser avec l'aide éventuelle de la direction des ressources humaines et du service juridique (entres autres). \\

Après analyse, le pilote de l'offre peut décider de décliner la proposition et de confirmer au client que l'opportunité ne sera pas saisie. Dans le cas contraire, une réponse positive sera donnée au client. Il faut alors créer et enregistrer le dossier auprès du secrétariat de maintenance, puis proposer des solutions avec le calcul du prix de revient. Enfin, il faudra choisir la solution la plus adaptée ainsi et les prix de vente retenus. Il faut ensuite rédiger l'offre initiale et la faire valider en interne, pour finalement la transmettre dans les délais impartis.


\subsection{Commande et revue de commande}

Après avoir effectué la revue de l'offre, les négociations avec le client sont effectuées. Une fois l'offre validée avec le client, le secrétariat de maintenance l'enregistre et diffuse un original et des copies du dossier de commande. Le responsable d'activité maintenance affecte ensuite la commande au porteur opérationnel en désignant un responsable. Une fois la revue de commande effectuée, le responsable d'affaire valide les données nouvelles après avoir réalisé un plan d'action de validation. Il faut ensuite négocier avec le client, et les deux parties pourront soit trouver un accord et accepter la commande, soit la refuser s'ils n'arrivent pas à trouver un terrain d'entente. Quelle que soit l'issue de la négociation, le service commercial référence le contrat et son état.

\subsection{Lancement des prestations de service et travaux}

Si la commande a été acceptée, il faut la lancer. Les spécifications client sont connues, et on dispose de toutes les données internes nécessaires, le responsable d'affaire peut alors prendre en compte le dossier contractuel et le dossier d'étude. Le RA va alors analyser les exigences et les besoins, et identifier les acteurs afin de faire valider l'organigramme. \\

Une réunion de lancement a alors lieu, réunion pour laquelle seront présents tous les acteurs majeurs du contrat : RA, responsable opérationnel contrat (ROC), service des méthodes, mais aussi éventuellement les commerciaux, le service de gestion, ou encore la direction des ressources humaines, parmi d'autres. Une fois la réunion effectuée, il faut mobiliser les ressources nécessaires pour les rendre disponibles et opérationnelles, avec l'aide indispensable du ROC. En ayant à disposition le dossier de synthèse et les spécifications client, le ROC va créer des procédures et documents opérationnels. Le service des méthodes va alors initialiser des systèmes de gestion. Le responsable d'affaires reprend ensuite la main pour faire le bilan sur tous les actions et documents réalisés jusqu'à présent. Enfin, il va prendre en charge la situation afin d'aboutir à une situation initiale connue et maîtrisée.

\subsection{Réalisation de prestations de maintenance}

La commande de prestations de services et travaux est bien en place : il faut maintenant les exécuter, conformément à la commande et aux documents amonts, sous la responsabilité du responsable activité maintenance. \\

Mais il est également important de gérer l'affaire. Si on dispose de la commande initiale et que tous les documents amonts sont à la disposition du responsable d'affaires, alors la gestion de l'affaire débute dans l'objectif d'obtenir la maîtrise économique et juridique du contrat et de prévoir à nouveau la fin d'affaire. Le RA doit ensuite identifier les avenants (cf. revue d'offre) et les travaux induits (cf. sous-processus "réalisation travaux induits"). \\

Enfin, il faut gérer les activités et le reporting. En ayant à sa disposition les données issues des différents SI, on peut suivre les activités et le reporting client, puis effectuer une revue périodique de contrat grâce à cela, pour se diriger vers les évolutions du contrat.

\subsection{Réalisation travaux induits}

À la suite d'une demande faite par le client ou un intervenant du contrat, on souhaite réaliser des prestations de maintenance et identifier les travaux induits. En étudiant le cahier des charges, le responsable d'affaire va, avec l'aval du responsable opérationnel contrat, décider de donner suite ou non. S'ils décident de ne pas donner suite, il faut en informer le client. \\

Sinon, ils vont ensuite décréter si les travaux sous tous induits. Si ce n'est pas le cas, ces derniers seront redirigés vers le processus Travaux pour que tout soit prêt. Sinon, le RA et le ROC vont se demander si le contrat inclut les modalités d'exécution des travaux induits. Deux cas se présentent alors : \\
\begin{description}
\item Le contrat n'inclut pas les modalités d'exécution. Les travaux sont alors réalisés sur devis. Le RA et le ROC, avec l'aide éventuelle d'autres entités de la société, vont alors directement chiffrer, valider et envoyer ledit devis signé en fonction des pouvoirs. Une fois la commande reçue, elle est validée. \\

\item Le contrat inclut bien des modalités d'exécution. Les travaux sont alors réalisés en dépenses contrôlées, ou bien sur devis (suivant les clauses contractuelles). Le chiffrage et la validation des charges travaux sont ensuite effectués, puis le client est informé, ou on lui envoie un devis. Après réception de la commande orale ou de l'ordre de service, le RA va valider ceci et l'enregistrer sur Supra. \\
\end{description}

La préparation des travaux débute sous la responsabilité du responsable opérationnel contrat. Les consignes d'exécution sont remises au responsable de l'exécution avec les informations sur les éventuels risques et les mesures à prendre. L'exécution des prestations est effectuée, aboutissant à la mise à jour des documents d'exécution. Le client signe alors les documents reçus (PV de réception, CRI\dots), ce qui entraîne le déclenchement de la facture auprès du service Marché, ainsi que la gestion de la garantie.

\subsection{Évolution du contrat}

Grâce à la disponibilité du tableau de bord affaire et activités, des données comptables du système Supra et des différentes orientations (interne et client), le contrat peut évoluer sous la responsabilité du responsable d'affaire. Les risques sont analysés et un bilan de l'affaire est effectué. La décision est ensuite prise de renouveller l'affaire sous sa forme initiale ou sous une autre forme.

\subsection{Solde de l'affaire et du contrat}

L'affaire touche à sa fin : grâce à son bilan, la revue de contrat, les commandes et les avenants, il faut en effectuer la revue. Le client a pu constater des écarts par rapport à ce qui était prévu. Le RA va alors décider de solder les prestations et travaux et d'en effectuer une recette. \\

Un état des lieux de sortie est éventuellement réalisé sous la responsabilité du responsable opérationnel contrat. À la suite de cette réalisation, des écarts peuvent éventuellement être constatés, menant à un plan d'action. Il faut donc traiter ces écarts, et les solder en levant les réserves. \\

Grâce à la recette des prestations travaux et contrat, le responsable d'affaire peut gérer la garantie et en fermer le compte, déterminant la fin de la période de garantie. L'affaire peut alors être soldée et archivée.

\section{Description des données}

\todo{Ajouter le tableau récapitulatif des données mise en jeu dans les processus}

\section{Structure organisationnelle}

\todo{Ajouter un organigramme ARIS}

