
Nous pouvons comparer les différents critères d’évaluation entre les deux solutions, afin d’identifier les avantages et inconvénients des solutions respectives. Cela permet par la suite d’adapter le choix de la solution aux besoins techniques et de mise en oeuvre de SPIE Sud-Est. Ainsi, nous pouvons remarquer dans le tableau ci-dessous que la solution spécifique dispose d’un score plus élevé que la solution standard, de façon générale. Cependant, ce score élevé requiert tout de même un fort investissement initial, des recommandations à prendre en compte sur le choix de la solution étant explicités dans le paragraphe suivant.

\begin{table}[H]
    \begin{tabular}{p{6cm}|p{2cm}|p{2cm}|p{3cm}|p{2cm}}
    Critère                             & \multicolumn{2}{l}{Solution standard} & \multicolumn{2}{l}{Solution spécifique} \\ \hline
    ~                                   & Évaluation & Score   & Évaluation   & Score   \\ \hline
    Mise en oeuvre                         & ~       & \bf{11} & ~            & \bf{14} \\ \hline
    Délai de mise en oeuvre                & 5 mois  & \bf{2}  & 12 mois      & \bf{1}  \\
    Délai d'adaptation au nouveau SI       & 1 mois  & \bf{4}  & 4 mois       & \bf{1}  \\ \hline
    Impact sur l'organisation              & ~       & \bf{4}  & ~            & \bf{9}  \\ \hline
    \it{Des structures}                    & Faible  & 1       & Moyen        & 2       \\
    \it{Des processus}                     & Moyen   & 2       & Léger        & 1       \\
    \it{De la relation partenaire}         & Faible  & 1       & Très positif & 3       \\ \hline
    Risques de mise en oeuvre              & Fort    & \bf{1}  & Faible       & \bf{3}  \\ \hline
    Critères techniques                    & ~       & \bf{6}  & ~            & \bf{10} \\ \hline
    Facilité d'intégration dans le SI      & Moyen   & \bf{1}  & Moyen        & \bf{1}  \\
    Adéquation aux besoins fonctionnels    & Bien    & \bf{1}  & Optimale     & \bf{3}  \\ \hline
    Qualités techniques                    & ~       & \bf{4}  & ~            & \bf{6}  \\ \hline
    \it{Fiabilité, sécurité}               & Moyen   & 2       & Faible       & \bf{1}  \\
    \it{Évolutivité, facilité de Maj}      & Bien    & 1       & Optimale     & \bf{3}  \\
    \it{Facilité d'utilisation, ergonomie} & Bien    & 1       & Moyenne      & \bf{2}  \\
    \end{tabular}
\end{table}