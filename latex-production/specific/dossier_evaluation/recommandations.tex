
Le tableau d’estimation des coûts nous permet d’analyser quelle solution serait à privilégier pour SPIE. Nous avons, pour rappel, d’un côté une solution standard qui repose sur SAP et d’un autre un solution nécessitant un développement interne. \\

La solution standard offre un outil prêt rapidement et facilement configurable. De plus, concernant le budget, les frais sont étalés sur le long terme et il ne faut pas un très gros investissement dès le début. \\

La solution spécifique, quant à elle, offre un outil permettant une évolution en adéquation avec les besoins de SPIE, en revanche, les frais sur le long terme sont moins importants que la solution standard mais un fort investissement est demandé au départ. \\

La direction de SPIE pourra donc choisir entre ces deux solutions qui présentent toutes deux leurs avantages et inconvénients. La stratégie de la direction à court et à long terme va en effet définir quelle solution est la mieux adaptée. \\

Si la priorité de la direction de SPIE est maximiser le ROI à court terme, alors nous lui conseillons qu’elle s’oriente vers notre solution élaborée à l’aide de SAP. En effet, cette solution a l’avantage d’offrir peu de délais dans la mise en place totale de la solution. On trouve également peu d’investissement à court terme. Enfin, et c’est un élément important à considérer, la maintenance du logiciel n’est pas à faire puisqu’elle sera sous-traitée. Ce point présente aussi avantages et inconvénients. Avantages, car a priori le fait d’utiliser une solution utilisée par de nombreuses entreprises permet de se prévenir de problèmes récurrents, mais il y a aussi des inconvénients car SPIE sera dépendant de l’expertise de SAP en cas de problème technique. \\

En revanche, si SPIE peut se permettre de fournir de plus gros investissement, et peut privilégier le ROI sur le long terme, alors le développement spécifique parait plus intéressant.  D’une part, il permet de faire évoluer constamment le produit en suivant précisément les besoins de SPIE, ce qui permet à l’ensemble des utilisateurs de gagner en confort d’utilisation. Ensuite, le retour sur investissement est plus important sur le long terme. Enfin, cette solution permet de ne pas dépendre d’une autre société qui pourrait faire monter le coût des licences ou arrêter le développement de son logiciel dans le futur. Ainsi, si les délais de SPIE sont plus souples, nous recommandons donc la solution spécifique.