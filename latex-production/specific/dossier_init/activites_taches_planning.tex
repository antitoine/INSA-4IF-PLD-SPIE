
\section{Identification des activités}

Les activités qui seront menées sur le projet présentées ci-après seront triées par l’objectif auquel elles appartiennent. Ces activités seront menées par les différents collaborateurs de l’équipe indépendemment    de leur rôle dans le projet. Seul le responsable de la tache aura le rôle le plaçant naturellement comme expert sur cette tâche. Il est important de noter qu’un responsable de tâche n’est pas nécessairement celui qui la réalise mais plutôt celui qui s’assure que cette tache sera menée à bien dans le temps imparti. Il est également de sa responsabilité de faire remonter toute information concernant un risque de glissement au niveau du planning. \\
  
Les activités sont donc les suivantes :\\

Activités de production : \\

\begin{description}
    \item[\textbullet] Recherche, Exploration
    \item[\textbullet] Analyse, Exploitation
    \item[\textbullet] Rédaction, Synthèse
    \item[\textbullet] Mise en forme
    \item[\textbullet] Packaging
\end{description}

Activités de gestion de la qualité : \\

\begin{description}
    \item[\textbullet] Vérification, Validation
    \item[\textbullet] Vérifier que les procédures mises en place sur le projet sont respectées
\end{description}

Activités de gestion de projet : \\

\begin{description}
    \item[\textbullet] Suivi des indicateurs, analyse et synthèse
    \item[\textbullet] Planification des tâches
    \item[\textbullet] Suivi de l’avancement des tâches
    \item[\textbullet] Remise des livrables
    \item[\textbullet] Assister aux réunions et animer les réunion hebdomadaires
    \item[\textbullet] Mettre en place des actions correctives
\end{description}

\section{Identification des tâches}

Les tâches présentées dans la suite de ce document ont été identifiées comme les tâches nécessaires à la réalisation des livrables demandés et seront donc triées par phase. Ensuite ces tâches seront affectées au type d’activité auquel elle sont liées.  

\subsection{Phase 1 - Initialisation}

\begin{description}
% -------------------------------------------------------------- P
    \item[] \bf{Tâches de production}
        \begin{description}
            \item[\textbullet] \it{Réaliser le dossier d’initialisation}
                \begin{description}
                    \item[\textbullet] Rédiger de la section “Objet du projet et contexte”
                    \item[\textbullet] Rédiger de la section “Livrables”
                    \item[\textbullet] Rédiger de la section “Mode opératoire et Outils choisis”
                    \item[\textbullet] Rédiger de la section “Organisation de l’équipe”
                    \item[\textbullet] Rédiger de la section “Gestion des Risques”
                \end{description}
            \item[\textbullet] \it{Réaliser le PAQ}
                \begin{description}
                    \item[\textbullet] Rédiger de la section “Gestion de la documentation du projet”
                    \item[\textbullet] Rédiger de la section “Cycle de vie des documents”
                    \item[\textbullet] Rédiger de la section “Procédures internes de validation”
                    \item[\textbullet] Rédiger de la section “Procédures de recette client“
                    \item[\textbullet] Rédiger de la section “Outils utilisés”
                \end{description}
            \item[\textbullet] \it{Mettre en forme le tableau de bord}
            \item[\textbullet] \it{Convertir le dossier d’initialisation vers LaTeX}
            \item[\textbullet] \it{Convertir le PAQ vers la LaTeX}
            \item[\textbullet] \it{Packaging des livrables} \\
        \end{description}
% -------------------------------------------------------------- Q
    \item[] \bf{Tâches de gestion de la qualité}
        \begin{description}
            \item[\textbullet] \it{Valider le dossier d’initialisation}
                \begin{description}
                    \item[\textbullet] Valider la section “Objet du projet et contexte”
                    \item[\textbullet] Valider de la section “Livrables”
                    \item[\textbullet] Valider de la section “Mode opératoire et Outils choisis”
                    \item[\textbullet] Valider de la section “Organisation de l’équipe”
                    \item[\textbullet] Valider de la section “Gestion des Risques”
                \end{description}
            \item[\textbullet] \it{Valider le PAQ}
                \begin{description}
                    \item[\textbullet] Valider de la section “Gestion de la documentation du projet”
                    \item[\textbullet] Valider de la section “Cycle de vie des documents”
                    \item[\textbullet] Valider de la section “Procédures internes de validation”
                    \item[\textbullet] Valider de la section “Procédures de recette client“
                    \item[\textbullet] Valider de la section “Outils utilisés”
                \end{description}
            \item[\textbullet] \it{Valider le tableau de bord}
            \item[\textbullet] \it{Relecture du dossier après conversion}
            \item[\textbullet] \it{Relecture du PAQ après conversion} \\
        \end{description}
% -------------------------------------------------------------- GP
    \item[] \bf{Tâches de gestion de projet}
        \begin{description}
            \item[\textbullet] Identifier les tâches
            \item[\textbullet] Faire le planning
            \item[\textbullet] Assister à la réunion CdP
            \item[\textbullet] Assister à la réunion RQ
            \item[\textbullet] Valider l'ébauche de Gantt
            \item[\textbullet] Animer la réunion hebdomadaire avec l’équipe
        \end{description}
\end{description}

\subsection{Phase 2 - Dossier d’expression du besoin}

\begin{description}
% -------------------------------------------------------------- P
    \item[] \bf{Tâches de production}
        \begin{description}
            \item[\textbullet] \it{Réalisation de l'Etude de l'Existant}
                \begin{description}
                    \item[\textbullet] Analyse des processus de gestion des contrats de maintenance
                        \begin{description}
                            \item[\textbullet] Analyser selon le point de vue fonctionnement
                            \item[\textbullet] Analyser selon le point de vue organisation
                            \item[\textbullet] Analyser selon le point de vue architectures informatiques
                        \end{description}
                    \item[\textbullet] Synthétiser les analyses
                        \begin{description}
                            \item[\textbullet] Rédiger la section "Gestion des contrats de maintenance, fonctionnement"
                            \item[\textbullet] Rédaction de la section "Gestion des contrats de maintenance, organisation"
                            \item[\textbullet] Rédaction de la section "Gestion des contrats de maintenance, architectures informatiques"
                        \end{description}
                    \item[\textbullet] Réaliser les modèles ARIS
                        \begin{description}
                            \item[\textbullet] Réaliser les modèles ARIS correspondant à la section "Gestion des contrats de maintenance, fonctionnement"
                            \item[\textbullet] Réaliser les modèles ARIS correspondant à la section "Gestion des contrats de maintenance, organisation"
                            \item[\textbullet] Réaliser les modèles ARIS correspondant à la section "Gestion des contrats de maintenance, architectures informatiques"
                        \end{description}
                    \item[\textbullet] Convertir le dossier d’Etude de l’Existant en LaTeX
                    \item[\textbullet] Produire le rapport ARIS
                \end{description}
            \item[\textbullet] \it{Effectuer le Benchmark}
                \begin{description}
                    \item[\textbullet] Effectuer les recherches sur Internet concernant les processus formalisés (SAP)
                    \item[\textbullet] Synthétiser les recherches
                    \item[\textbullet] Conversion le document de synthèse en LaTeX
                \end{description}
            \item[\textbullet] \it{Spécifier le SI cible}
                \begin{description}
                    \item[\textbullet] Réaliser la modélisation des données
                    \item[\textbullet] Réaliser la modélisation des différents processus
                    \item[\textbullet] Rédiger le document de synthèse des modèles
                    \item[\textbullet] Rédiger du document de synthèse du modèle de données
                \end{description}
            \item[\textbullet] \it{Fusion des documents}
            \item[\textbullet] \it{Packaging des livrables}
        \end{description}
% -------------------------------------------------------------- Q
    \item[] \bf{Tâches de gestion de la qualité}
        \begin{description}
            \item[\textbullet] \it{Valider l’Etude de l’Existant}
                \begin{description}
                    \item[\textbullet] Valider la section "Gestion des contrats de maintenance, fonctionnement"
                    \item[\textbullet] Valider de la section "Gestion des contrats de maintenance, organisation"
                    \item[\textbullet] Valider de la section "Gestion des contrats de maintenance, architectures informatiques"
                \end{description}
            \item[\textbullet] \it{Valider les modèles ARIS}
                \begin{description}
                    \item[\textbullet] Valider les modèles ARIS correspondant à la section "Gestion des contrats de maintenance, fonctionnement"
                    \item[\textbullet] Valider les modèles ARIS correspondant à la section "Gestion des contrats de maintenance, organisation"
                    \item[\textbullet] Valider les modèles ARIS correspondant à la section "Gestion des contrats de maintenance, architectures informatiques"
                \end{description}
            \item[\textbullet] \it{Valider le Benchmark}
            \item[\textbullet] \it{Valider la spécification du SI cible}
                \begin{description}
                    \item[\textbullet] Valider la modélisation des données
                    \item[\textbullet] Valider la modélisation des différents processus
                    \item[\textbullet] Valider le document de synthèse des modèles
                    \item[\textbullet] Valider du document de synthèse du modèle de données
                \end{description}
            \item[\textbullet] \it{Relecture du l’Etude de l’Existant après conversion}
            \item[\textbullet] \it{Relecture des modèles ARIS après conversion}
            \item[\textbullet] \it{Relecture du Benchmark}
            \item[\textbullet] \it{Relecture de la spécification du SI cible}
        \end{description}
% -------------------------------------------------------------- GP
    \item[] \bf{Tâches de gestion de projet}
        \begin{description}
            \item[\textbullet] Animer les réunions hebdomadaires
            \item[\textbullet] Suivre les indicateurs
            \item[\textbullet] Entreprendre des actions correctives si nécessaire 
            \item[\textbullet] Rédiger les documents de suivi
        \end{description}
\end{description}

\subsection{Phase 3 - Dossier d'élaboration de solution}

\begin{description}
% -------------------------------------------------------------- P
    \item[] \bf{Tâches de production}
        \begin{description}
            \item[\textbullet] \it{Réaliser le dossier d'Elaboration de Solutions}
                \begin{description}
                    \item[\textbullet] Réaliser l'Etude de la solution spécifique
                        \begin{description}
                            \item[\textbullet] Concevoir la solution informatique
                            \item[\textbullet] Concevoir l’architecture applicative
                                \begin{description}
                                    \item[\textbullet] Lister les blocs applicatifs
                                    \item[\textbullet] Décrire les blocs : outils/service et données
                                    \item[\textbullet] Décrire les échange de données entre les blocs
                                    \item[\textbullet] Modéliser l’architecture d’exécution
                                    \item[\textbullet] Réaliser le schéma général
                                \end{description}
                    \item[\textbullet] Concevoir l’architecture technique
                        \begin{description}
                            \item[\textbullet] Lister les éléments actifs : réseau, serveurs, postes de travail
                            \item[\textbullet] Concevoir l’architecture logicielle
                        \end{description}
                    \item[\textbullet] Identifier les nouvelles données si nécessaire
                    \item[\textbullet] Concevoir la solution organisationnelle
                        \begin{description}
                            \item[\textbullet] Identifier les acteurs humains et leurs rôles
                            \item[\textbullet] Identifier les flux d'information
                        \end{description}
                    \item[\textbullet] Concevoir la solution fonctionnelle
                        \begin{description}
                            \item[\textbullet] Identifier les nouveaux processus
                        \end{description}
                \end{description}
                    \item[\textbullet] Réaliser l'Etude de la solution ERP
                        \begin{description}
                    \item[\textbullet] Réaliser le rapport ARIS
                        \begin{description}
                            \item[\textbullet] Rédiger de l'introduction
                            \item[\textbullet] Réaliser la vue globale (DCPV) d'un périmètre de projet en terme des objectifs SPIE et des processus SAP ByD pertinents
                            \item[\textbullet] Réaliser la vue organisationnelle 
                            \item[\textbullet] Réaliser la vue informationnelle macro
                            \item[\textbullet] Réaliser la vue fonctionnelle (CPE) sur certaines fonctions dont l'alignement nécessite une explication
                            \item[\textbullet] Rédiger le glossaire 
                        \end{description}
                \end{description}
                \end{description}
            \item[\textbullet] \it{Packager le dossier d'Elaboration de Solutions}
                \begin{description}
                    \item[\textbullet] Fusionner des rapports intermédiaires
                \end{description}
        \end{description}
% -------------------------------------------------------------- Q
    \item[] \bf{Tâches de gestion de la qualité}
        \begin{description}
            \item[\textbullet] \it{Valider l'Etude de la solution spécifique}
                \begin{description}
                    \item[\textbullet] Valider la solution informatique
                        \begin{description}
                            \item[\textbullet] Valider l'architecture applicative
                            \item[\textbullet] Valider l'architecture technique
                            \item[\textbullet] Valider les données identifiées
                        \end{description}
                    \item[\textbullet] Valider la solution organisationnelle
                    \item[\textbullet] Valider la solution fonctionnelle
                \end{description}
            \item[\textbullet] \it{Valider l'Etude de la solution ERP}
            \item[\textbullet] \it{Valider du rapport ARIS}
            \item[\textbullet] \it{Valider le dossier complet}
        \end{description}
% -------------------------------------------------------------- GP
    \item[] \bf{Tâches de gestion de projet}
        \begin{description}
            \item[\textbullet] Animer les réunions hebdomadaires
            \item[\textbullet] Suivre les indicateurs
            \item[\textbullet] Entreprendre des actions correctives si nécessaire 
            \item[\textbullet] Rédiger les documents de suivi
        \end{description}
\end{description}

\subsection{Phase 4 - Réalisation du dossier de choix}

\begin{description}
% -------------------------------------------------------------- P
    \item[] \bf{Tâches de production}
        \begin{description}
            \item[\textbullet] \it{Décrire de la solution spécifique}
                \begin{description}
                    \item[\textbullet] Rappeler les fonctionnalités de la solution informatique et organisationnelle
                    \item[\textbullet] Chiffrer les coûts (d'acquisition et de possession)
                    \item[\textbullet] Évaluer le retour sur investissement (gains)
                    \item[\textbullet] Évaluer les autres critères de comparaison
                \end{description}
            \item[\textbullet] \it{Décrire la solution ERP}
                \begin{description}
                    \item[\textbullet] Rappeler les fonctionnalités de la solution informatique et organisationnelle
                    \item[\textbullet] Chiffrer les coûts (d'acquisition et de possession)
                    \item[\textbullet] Évaluer le retour sur investissement (gains)
                    \item[\textbullet] Évaluer les autres critères de comparaison
                \end{description}
            \item[\textbullet] \it{Faire le tableau comparatif et pondération}
                \begin{description}
                    \item[\textbullet] Réaliser le tableau comparatif
                        \begin{description}
                            \item[\textbullet] Définir des critères de comparaison
                            \item[\textbullet] Remplir du tableau
                            \item[\textbullet] Rédiger de la synthèse du tableau
                            \item[\textbullet] Effectuer la recette du tableau
                        \end{description}
                \end{description}
        \end{description}
% -------------------------------------------------------------- Q
    \item[] \bf{Tâches de gestion de la qualité}
        \begin{description}
            \item[\textbullet] Valider la description de la solution spécifique
            \item[\textbullet] Valider la description de la solution ERP
            \item[\textbullet] Valider le tableau
        \end{description}
% -------------------------------------------------------------- GP
    \item[] \bf{Tâches de gestion de projet}
        \begin{description}
            \item[\textbullet] Animer les réunions hebdomadaires
            \item[\textbullet] Suivre les indicateurs
            \item[\textbullet] Entreprendre des actions correctives si nécessaire 
            \item[\textbullet] Rédiger les documents de suivi
        \end{description}
\end{description}

\subsection{Phase 5 - Préparation de la présentation}

\begin{description}
% -------------------------------------------------------------- P
    \item[] \bf{Tâches de production}
        \begin{description}
            \item[\textbullet] Préparer la présentation
            \item[\textbullet] Rédiger le dossier bilan
        \end{description}
\end{description}

\bf{Note concernant la plannification :} les deadlines et les temps alloués à chaque tâche sont décrits de manière exhaustive dans le Tableau de Bord concernant la page de suivi des tâches. Veuillez vous reporter à ce document pour tout complément d’information concernant l’aspect temporel des tâches identifiées précédemment et leur ordonnancement.
