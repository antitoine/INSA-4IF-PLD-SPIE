\section{Objet du projet}

À l'origine de ce projet, une entreprise spécialisée dans les domaines de l'énergie, de la mécanique et des réseaux de communication : Spie. Son souhait ? Rendre son activité de « gestion des contrats de maintenance » plus homogène, afin que tous les aspects de son métier respectent un processus similaire. Mais le besoin de Spie ne s'arrête pas ici : la société s'oriente de plus en plus vers le secteur des énergies vertes, et ce domaine commence à faire partie intégrante de l'entreprise. Il faut donc qu'ici aussi le processus de maintenance des équipements industriels liés à ce nouveau secteur soit homogène à tous les autres processus de la société, tout comme il faut que n'importe quel nouveau secteur auquel Spie souhaiterait se consacrer soit un besoin facilement intégrable. Notre projet doit donc s'inscrire dans une démarche de globalisation, dans le but de rendre les activités de la société moins hétérogènes. \\
    
Sur un marché où tout va toujours plus vite, une telle demande d'homogénéisation ne semble pas incongrue, puisque cela permettra à Spie d'améliorer grandement son efficacité dans toute l'étape de maintenance des équipements. Cela facilitera également la compréhension des intervenants puisque les processus seront plus simples à appréhender et également similaires les uns aux autres : les utilisateurs seront donc beaucoup plus polyvalents.

\section{Contexte du projet}

Notre étude se positionne en réalité dans un projet de bien plus grande ampleur : en effet, nous nous limiterons ici à l'étude préalable, qui en est la toute première étape. L'objectif final de ce projet sera de proposer deux solutions permettant d'améliorer les processus de maintenance, qui devront impérativement prendre en compte les retours d'expérience que nous avons pu recevoir. Nous ne nous intéresserons pas à la suite de ce projet, mais allons plutôt nous assurer que nous exprimons les besoins de manière fonctionnelle et non en terme de solutions. L'analyse ainsi effectuée nous permettra de dégager toutes les fonctionnalités nécessaires à la réalisation d'un projet futur, en mettant au point un document définissant fonctionnellement le besoin, indépendamment d'une solution technique. \\
    
Pour ce faire, nous allons mettre en oeuvre certaines techniques de production, incluant la spécification de solutions informatiques, la mise en place de méthodes de conception ou encore l’élaboration d’un Plan d’Assurance Qualité. En constituant des équipes dont les rôles de chacun sont bien définis et répartis selon les compétences de chaque personne, nous pourrons organiser notre projet de la meilleure façon possible et proposer le suivi de l’avancement du projet en temps réel.

