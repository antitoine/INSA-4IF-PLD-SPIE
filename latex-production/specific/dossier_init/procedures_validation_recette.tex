
\section{Renvoi au PAQ}

Les procédures de validation et de recette résumées ci-dessous sont décrites de manière détaillée dans le PAQ référencé PAQ/4401/1.

\section{Procédure interne de validation}

Lorsque le document atteint cette étape, il a déjà vécu l'ensemble du processus de production c'està dire sa création sous la forme d'un document collaboratif Google Document et il a été converti au format LaTeX pour être produit dans sa forme finale. 

Cette étape permet de garantir une dernière fois la bonne mise en forme du document, tout en déclarant le document terminé après avoir réalisé une dernière relecture, empêchant ainsi toute retouche par la suite. Cette étape est réalisée conjointement par le Responsable Qualité et par le Chef de Projet. Le document devient alors un document livrable, au format PDF, pouvant être livré au client et archivé.

\section{Procédure de recette externe}

Lorsqu’un document est considéré comme étant livrable, ce dernier doit être soumis au client. Afin d’attester ladite livraison à une date donnée, prouvant ainsi le respect des délais, un procès-verbal de livraison est édité. Ainsi, le chef de projet modifie le document type disponible en annexe XX, adaptant ce dernier au livrable devant être fourni au client. Les informations importantes à mentionner dans ces procès-verbaux de livraison sont les suivantes : \\

\begin{itemize}
    \item[\textbullet] Référence et intitulé du livrable
    \item[\textbullet] Récepteur du livrable : nom et signature
    \item[\textbullet] Émetteur du livrable : nom et signature
    \item[\textbullet] Date de la livraison \\
\end{itemize}
    
Ces procès-verbaux de livraison sont produits en double exemplaires, permettant ainsi à l’émetteur et au récepteur du livrable de posséder une attestation de remise. Les deux parties s’engagent ainsi, par leur signature, à clôturer le cycle de vie du document livré, ce dernier ne pouvant être retouché par la suite.
