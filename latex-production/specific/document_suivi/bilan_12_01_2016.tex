
\section{Bilan d'avancement}

Le retard accumulé ayant provoqué un glissement relativement important dans le planning des tâches s'est résorbé grâce au travail efficace, hors séance, de l'équipe. Nous avons donc réussi à livrer l'ensemble des documents constituant le dossier bilan de la phase d'Expression des Besoins dans les temps. Nous pouvons donc maintenant nous consacrer pleinement à la réalisation des tâches de la phase suivante qui consiste à élaborer deux solutions : une spécifique et une standard basée sur l'utilisation de l'ERP SAP. \\

Les tâches relevant de la qualité se sont diversifiées et comprennent lors de cette courte période la relecture et l'intégration des documents afin de certifier la qualité des livrables qui ont été rendus. Mais également à vérifier le bon usage des outils et notamment du répertoire partagé dans lequel quelques abus ont été constatés et ont rapidement été rectifiés. Les livrables ont été validés et livrés dans les temps ce qui tend à démontrer l'efficacité des processus mis en place dans le PAQ en début de projet malgré les évènements extérieurs ayant quelque peu perturbé le déroulment du projet.\\

Concernant la gestion de projet, les livrables ayant été remis dans les temps malgré un retard conséquent constitue un évènement important dans la vie du projet et justifie les performances des membres de l'équipe et leur efficacité tant sur le plan de l'investissement personnel que du travail de groupe dans lequel aucune tension n'est à déplorer. L'ambiance de travail est saine et l'équipe bien que peu motivée par le projet réalise le travail attendu et comprend de plus en plus l'intérêt de ce dernier au fur et à mesure que le projet avance. Après la validation des livrables, les tâches de gestion de projet vont maintenant se concentrer sur la revue du planning des tâches et la pertinence des tâches définies en début de projet. Il sera également nécessaire de continuer à effectuer des points de synchronisation réguliers qui permettent à l'équipe de se situer.

\section{Bilan humain}

Il semblerait que la période des fêtes n'ait pas été propice au repos des membres de l'équipe qui semblent encore fatigués mais il est possible de sentir un envie commune d'avancer. Le moral n'est pas au plus haut mais nous constatons une certaine stagnation de cet indicateur ce qui signifie que bien que l'évolution ne soit pas positive elle n'empire pas non plus. Tous les membres de l'équipe sont prêts à attaquer la nouvelle phase avec l'envie d'éviter un nouveau glissement. Même si certains membres de l'équipe ne sont pas du tout intéressés par le projet, ils prennent leur responsabilité et effectuent le travail qui leur est demandé ce qui est appréciable. L'ambiance est bonne au sein du groupe.
