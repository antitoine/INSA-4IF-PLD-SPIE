
\section{Bilan d'avancement}

Malgré le nom donné à cette phase, nous continuons à travailler en deux groupes. Nous progressons dans l'élaboration de la solution spécifique, après avoir terminé la description des cas d'utilisation nous séparons le travail pour développer les différents aspects de la solution en parallèles : fonctionnalités, gestion des données ou encore intégration avec l'existant. La seconde équipe continue quant à elle de réaliser les différents modèles ARIS nécessaires à la mise en place de la solution SAP dite standard. \\

Les tâches de la Qualité se limitent à la relecture et l'intégration des documents. Elles consistent aussi à vérifier le bon usage des outils. \\

Concernant la gestion de projet, une réunion de synchronisation à ouvert la séance et a été suivie d'une mse à jour du tableau de bord. Les tâches diffèrent légèrement de la réalité du projet ce qui était envisageable étant donné que celles-ci ont été définies lors de la phase d'initialisation du projet. Le suivi des indicateurs du tableau de bord annonce une confirmation de la hausse du moral de l'équipe. La rédaction du bilan de la séance précédente a été repoussée à la séance suivante par manque de temps. En effet, le Chef de Projet s'investit aussi dans des tâches de production.

\section{Bilan humain}

La hausse précédemment relevée concernant le moral de l'équipe se confirme et nous surveillerons cet indicateur pour vérifier que ce dernier ne se dégrade pas. L'équipe continue d'avancer sans poser de problème. La motivation n'est pas toujours la mais les membres de l'équipe se motivent à tour de rôle pour réaliser leurs tâches et assument bien leurs responsabilités. Il est agréable de travailler dans ce cadre où les différents collaborateur font preuve de sérieux.  
