
\section{Bilan d'avancement}

Les tâches concernant l'étude de l'existant ont été partiellement réaisées et se sont révélées plus longues, en effet, l'analyse des différents processus prends un certain temps et la plateforme ARIS a été inaccessible la plupart du temps pendant les séance de projet. Cette même plateforme n'est pas accessible à l'extérieur de l'INSA et ne peut donc être utilisée que dans les salles de TP. Les causes de ce problème sont identifiées mais, de nouveau, sont indépendantes de notre volonté. Un seul membre de l'hexanome est parvenu, après plus de 2h de bataille, à faire fonctionner l'application dans un environnement technique particulier ne permettant pas à tous les membres de l'hexanome d'en faire de même (enivronnement d'exécution propriétaire). \\

Concernant la qualité, les tâches se limitent à vérifier le bon usage des outils et la qualité du contenu et des recherches effectuées. Aucun livrable n'a encore été constitué. \\

La gestion de projet, quant à elle, consiste à réaliser le suivi des indicateurs et animer les réunions. Il est également nécessaire de mettre à jour la liste des tâches et l'avancement de ces dernières. La vérification du bon usage du tableau de bord, notamment la saisie des temps, est aussi une des activités exercée lors des séances de projet. La rédaction de ce bilan hebdomadaire permet de prendre du recul sur le travail accompli et le travail restant. \\

\section{Bilan humain}

Vu le peu d'amélioration concernant les conditions d'accès aux outils de travail, le moral continue de chuter. Espérons que cette situation ait trouvée une réponse après les vacances et que ces dernières aient permis aux membres de l'équipe de se ressourcer pour reprendre le projet dans les meilleures conditions. Il est également difficile en tant que chef de projet de faire en sorte que l'équipe se sentent concernée par le projet si les tâches qui leurs sont affectées ne peuvent être réalisées.
