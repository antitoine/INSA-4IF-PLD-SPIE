
\section{Bilan d'avancement}

Concernant les tâches de production, nous avons fait le choix de séparer le groupe en deux afin de développer plus rapidement les deux solutions. Nous avons donc commencé à travailler sur les deux solutions et à l'issue de cette séance, les deux solutions sont initialisées. Le développement de la solution spécifique a atteint la phase de conception des cas d'utilisation. Tandis que la seconde équipe est parvenue à mettre en \oe{}uvre la génération d'un rapport ARIS customisé pour la solution standard et a développé les premiers modèles constituant cette solution. \\

La production de livrables n'étant pas à l'ordre du jour, les tâches de la Qualité consistent principalement en la relecture et l'intégration des documents. Elles consistent toujours à vérifier le bon usage des outils. \\

Concernant la gestion de projet, la phase a commencé par la tenue d'une réunion comme à chaque séance pour faire un point sur ce qui a été réalisé et ce qui doit l'être, le but étant au final de guider un maximum les membres de l'équipe pour que ceux-ci aient des objectifs clairs et puissent mener à bien les tâches qui leur sont confiées. La rédaction du bilan de la séance précédente et le suivi des ndicateurs du tableau de bord font également partie des tâches qui ont été réalisées dans le cadre de la gestion de projet. 

\section{Bilan humain}

Nous notons une légère hausse dans le moral de l'équipe, cette hausse s'explique en partie par l'intérêt que portent les membres à la phase qui vient de débuter. En effet, la majorité des membres de l'équipe sont naturellement attirés par les aspects techniques et la conception de solutions en général. Il est donc naturel de constater une hausse du moral, il sera intéressant de surveiller cet indicateur afin de vérifier que cette hausse se confirme. 
