
\section{Bilan d'avancement}

Au terme de cette séance d'initialisation, l'ensemble des tâches a été réalisé dans les temps. Il n'y a donc pas, à ce stade, de glissement concernant le planning. L'ajout des exercices ARIS à quelque peu perturbé le programme mais cette perturbation à été très bien absorbé par le travail efficace mené par les membres de l'équipe. Il a cependant été décidé de mettre de côté certains exercices pour dégager conserver du temps pour réaliser les tâches que l'on pourrait qualifier de principales. Les indicateurs concernant les livrables sont donc verts. \\

Du point de vue de la gestion de la qualité, la vérification de la bonne application des règles énoncées dans le PAQ est réalisée à chaque séance sous la forme de revues des livrables et des différents documents en production. Ces tâches se révèlent utiles dans le sens où l'on constate parfois quelques dérives au niveau de l'utilisation des outils collaboratifs qui pourraient conduire à un certain désordre si ces vérifications n'étaient pas effectuées. \\

Pour finir, les tâches de gestion de projet sont menées régulièrement afin de guider, au mieux, les membres de l'équipe. La définition des tâches n'est pas une tâche facile et il est difficile d'atteindre une granularité suffisante sur des tâches pour lesquelles la visibilité n'est pas bonne (tâches temporellement lointaine et/ou floue). Un travail de mise à jour des tâches est donc effectué la veille de chaque séance afin d'assurer une définition suffisante des tâches pour que les membres n'aient pas à se poser de question et puissent se contenter de réaliser ou de suivre la réalisation des tâches qui sont sous leur responsabilité. \\

\section{Bilan humain}

D'un point de vue humain on note une baisse dans le moral des membres de l'équipe il est en effet difficile de motiver les membres qui sont pour la plupart déçus des problèmes, récurrents, d'accès aux ressources. De même, ils estiment que la durée de travail recommandée en dehors des heures de projet est largement sous-estimée. \\
Il est malheureusement impossible d'envisager des actions correctives car ces problèmes ne peuvent pas être résolu par nos soins.
