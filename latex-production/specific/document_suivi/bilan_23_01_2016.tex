
\section{Bilan d'avancement}

Les tâches de production ont été réparties en début de séance concernant l'élaboration des solutions. L'avance sur le planning de la phase précédente a permis d'anticiper la réalisation de certaines tâches telles que le chiffrage des coût des différentes solution. Une partie des membres de l'équipe se consacre donc, à partir de cette séance, sur la réalisation du dossier de choix et commiunique avec l'autre partie de l'équipe qui se concentre sur la préparation de la présentation.\\

Les tâches de la qualité se concentrent autour de la vérification du respect des produre et du contrôle des données employées pour chiffrer. En effet, les sources doivent être claires et fiables pour donner une approximation la plus juste possible du coût que pourrait avoir la mise en place de ces solutions.\\

Les tâches de gestion de projet sont assez réduite désormais, l'animation d'une réunion en début de séance suivi de la rédaction du bilan hebdomadaire et enfin du bilan final du projet. Les tâches sont définitives et le chef de projet participe à la réalisation de ces dernières.   

\section{Bilan humain}

L'équipe est de nouveau au complet et prête, à en croire l'évolution de l'humeur, à travailler afin de clore le projet dans les temps. Il est clair qu'il reste des tâches à accomplir mais l'optimisme du groupe montre la volonté de réaliser ces dernières. Aucun évènement notable n'est a rapporter.
