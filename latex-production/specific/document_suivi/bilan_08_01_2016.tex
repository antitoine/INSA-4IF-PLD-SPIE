
\section{Bilan d'avancement}

La poursuite de la réalisation des tâches de l'étude de l'existant est effectuée en paralèlle de l'initialisation des taches de la phase de benchmarking. Le benchmarking nécessite selon nos estimations moins de temps de travail et le déséquilibre entre l'étude de l'existant et le benchmarking en terme de charge de travail se ressent par un léger glissement sur le planning. Ce glissement n'est pas important mais nous apporterons une attention particulière afin de veiller à ce que celui-ci ne s'aggrave pas. Nous serons dans les temps pour fournir les livrables attendus.

Des facteurs extérieurs ayant décalé la réalisation de l'exercice SAP par l'équipe celui-ci n'a pu être réalisé pendant cette phase et a été rendu au moment de la reprise du projet après la période des fêtes. La date de rendu n'ayant pas été modifiée, le document est considéré comme étant rendu en retard ce qui n'est pas de notre fait, nous tenons à le préciser.

Concernant la qualité, les tâches se limitent à vérifier le bon usage des outils et la qualité du contenu et des recherches effectuées. Des livrables sont en cours de constitution et leur constitution progressive facilite le travail de releecture par l'intégration progressive. \\

La gestion de projet, quant à elle, consiste à réaliser le suivi des indicateurs et animer les réunions. Des efforts conséquents ont été fourni afin de guider au mieux les membres de l'équipe dans leur travail afin de garantir une certaines efficacité pour tenter de compenser le glissement occasionné par l'indisponibilité de certaines ressources essentielles pour la réalisation de l'étude. Il est, comme d'ordinaire, toujours nécessaire de mettre à jour la liste des tâches et l'avancement de ces dernières. Le tableau de bord est surveillé et mis à jour lors des séances de projet et des contrôles de cohérence sont effectués à chaque séance. Ce bilan est rédigé comme d'ordinaire sur le temps de la séance. \\

\section{Bilan humain}

L'accès au ressources n'est toujours pas rétablit mais certaines tâches peuvent tout de même être menées. L'équipe fait des efforts et fournit un travail important comparé à la charge anoncée initialement. Tout n'est pas dimmensionné à la perfection et ce malgré le moral. Nous esperons que les conditions s'améliorerons après la période des fêtes. Ce bilan étant rédigé après la période précédement citée il est d'ors et déjà possible de dire que l'accès au ressources semble être rétablit et stable mais les congés n'ont pas fourni le repos nécessaire à l'équipe.
