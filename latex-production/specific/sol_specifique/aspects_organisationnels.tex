\section{Modifications des acteurs}

Si le déploiement d’une solution spécifique n’implique pas de changements majeurs dans l’organisation de la société, la mise en place d’une base de connaissances nécessitera tout de même de créer un nouveau poste de travail. Ce dernier correspond à un gestionnaire de base de connaissances, qui aura pour objectif de maintenir et surveiller le bon fonctionnement de la base. \\

La mise en place de l’axe d’amélioration porté sur la performance va également nécessiter de nouvelles créations de poste. En effet, cet axe d’amélioration se concentre autour de la mise en place d’indicateurs de performance, permettant de mesurer la santé de la société à travers une mesure régulière et normalisée de ses activités. Un responsable performance, en charge de la surveillance de la bonne utilisation de ces indicateurs, est de fait requis. Celui-ci aura également pour mission de créer de potentiels nouveaux indicateurs, qui pourront être par la suite intégrés au module applicatif “Outil PERF”. De moindre nécessité, mais ayant tout de même une importance, des postes de travail de type analystes de performance pourront être ajouté sous la direction du responsable performance. Ces analystes auront pour rôle de réaliser des rapports en prenant en considération les différents indicateurs de performance, de manière plus détaillée que les rapides analyses effectuées par les commerciaux ou les responsables d’affaires qui ont seulement besoin d’un aperçu global des indicateurs. Les analystes auront alors pour tâche de détecter en amont d’éventuels problèmes liés aux activités de la société. \\

Enfin, la mise en place de l’axe d’amélioration porté sur la gestion avancée des risques nécessite également un nouvel acteur. Ce nouvel acteur correspond à un responsable des risques, chargé de surveiller la bonne utilisation des indicateurs de risque, tout en améliorant leur bonne utilisation afin de faire baisser davantage les niveaux de risque.

\section{Modifications des tâches}

Si la mise en oeuvre de cette solution spécifique ne modifie pas fondamentalement l’organisation de SPIE Sud-Est, les postes existants doivent cependant adapter légèrement leurs procédures de travail et leurs tâches. \\

Ce changement est notamment applicable aux acteurs impactés principalement par la mise en place d’un mécanisme de retour d’expériences avancé, soit les techniciens. En effet, ces derniers seront équipés de terminaux mobiles, leur permettant d’accéder aux données dont ils ont besoin leur d’une prestation de maintenance, directement sur le terrain. Cela implique cependant un changement dans leurs habitudes de travail, puisqu’il faudra dans un premier temps s’habituer à l’application mobile mise à leur disposition, mais également au nouvel enchaînement des tâches, le retour d’expérience devant être saisi à l’instant T, et non plus lors du retour sur les locaux de SPIE Sud-Est ou sur des formulaires papier. \\

De même, la mise à disposition de la base de connaissances, permettant de diminuer les risques lors des prestations de maintenance ou lors de la rencontre d’un problème connu, implique également un changement dans les processus de travail des utilisateurs. Un nouveau réflexe, celui de rechercher dans la base de connaissances le problème potentiellement déjà rencontré, doit être acquis par les employés de SPIE Sud-Est.
