% --------------------------- REX
\section{Retour d'Expérience (REX)} % --------------------------- REX
% --------------------------- REX

\subsection{Spécifications fonctionnelles}

\subsubsection{Cas d’utilisations}

% #############################################
\noindent\textsc{\bf{A1 - UC1 :} Consulter une prestation de maintenance}
\begin{shaded}
\noindent\textsc{Résumé :}\\

Pendant une intervention, le technicien souhaite accède à l’historique du dossier pour une prestation de maintenance donné. \\

\noindent\textsc{Utilisateur :}\\

Technicien ou Responsable d’affaires ou Commercial \\

\noindent\textsc{Scénario :} \\
\begin{enumerate}
    \item   L’utilisateur fait une recherche avec la référence d’un contrat ou le nom d’une société
    \item   Le système affiche la liste des contrats pouvant correspondre
    \item   L’utilisateur sélectionne son contrat
    \item   Le système affiche l’état actuel du contrat avec la liste des commandes
    \item   L’utilisateur sélectionne la commande correspondant à son intervention
    \item   Le système affiche l’état actuel de la commande avec la liste des prestations de maintenance
    \item   L’utilisateur sélectionne la prestation de maintenance correspondant à l’intervention
    \item   Le système retourne la prestation de maintenance et les pièces jointes qui lui sont associées \\
\end{enumerate}
\noindent\textsc{Scénarii alternatifs :} \\
\begin{enumerate}
    \item L’utilisateur fait une recherche avec la référence de la commande
    \begin{enumerate}
        \item Le système affiche la liste des commandes pouvant correspondre à la recherche
        \item On reprend le scénario nominal à l’étape de 5 \\
    \end{enumerate}
    \item L’utilisateur fait une recherche avec la référence de la prestation de maintenance
    \begin{enumerate}
        \item Le système affiche la liste des prestations de maintenance pouvant correspondre à la recherche
        \item On reprend le scénario nominal à l’étape 7
    \end{enumerate}
\end{enumerate}
\end{shaded}
% #############################################
\noindent\textsc{\bf{A1 - UC2 :} Saisir le retour d’expérience}
\begin{shaded}
\noindent\textsc{Résumé :}\\

Après une prestation de maintenance effectuée par un technicien, il peut rentrer un résumé et les détails sur cette dernière dans une base de connaissance. \\

\noindent\textsc{Utilisateur :} \\

Technicien \\

\noindent\textsc{Scénario :} \\
\begin{enumerate}
    \item L’utilisateur consulte une prestation de maintenance (A1 - UC1) et demande une saisie de retour d’expérience
    \item Le système demande à l’utilisateur de saisir les informations sur son retour d’expérience
    \item L’utilisateur renseigne les différents champs et valide la saisie
    \item Le système l’enregistre et affiche le résultat de sa saisie \\
\end{enumerate}
\end{shaded}
% #############################################
\noindent\textsc{\bf{A1 - UC3 :} Associer des pièces jointes au retour d’expérience}
\begin{shaded}
\noindent\textsc{Résumé :}\\

L’utilisateur veut associer des pièces jointes de différentes catégories à la prestation de maintenance. \\

\noindent\textsc{Utilisateur :} \\

Technicien \\

\noindent\textsc{Scénario :} \\
\begin{enumerate}
    \item L’utilisateur consulte une prestation de maintenance (A1 - UC1) et demande l’ajout de pièces jointes
    \item L’utilisateur choisit les pièces jointes qu’il veut associer au dossier et leur affecte une catégorie
    \item Les pièces jointes sont récupérées depuis le média qu’il utilise (tablette, ordinateur, etc.)
\end{enumerate}

\end{shaded}

\subsubsection{Services}

\todo{Compléter cette section}

\subsubsection{Identification du bloc applicatif}

\todo{Compléter cette section}

\subsection{Gestion des données}

\subsubsection{Objets métiers}

\todo{Compléter cette section}

\subsubsection{Représentation de la donnée}

\todo{Compléter cette section}

\subsection{Infrastructure}

\subsubsection{Solution retenue}

\todo{Compléter cette section}

\subsubsection{Déploiement de la solution}

\todo{Compléter cette section}

% --------------------------- PERF
\section{Analyse des Performances (PERF)}% --------------------------- PERF
% --------------------------- PERF

\subsection{Spécifications fonctionnelles}

\subsubsection{Cas d’utilisations}

\noindent\textsc{\bf{A2 - UC1 :} Calculer le bénéfice d’un contrat}
\begin{shaded}
\noindent\textsc{Résumé :}\\

L’utilisateur consulte le dossier d’un contrat puis demande l’estimation du bénéfice associé à celui-ci, tout en estimant les potentiels risques, pouvant survenir lors de l’exécution du contrat. \\

\noindent\textsc{Utilisateur :} \\

Commercial \\

\noindent\textsc{Scénario :} \\
\begin{enumerate}
    \item L’utilisateur recherche le contrat par sa référence ou le nom de la société dans laquelle il intervient
    \item Le système affiche la liste des contrats pouvant correspondre à la recherche
    \item L’utilisateur sélectionne son contrat
    \item Le système affiche l’état actuel du contrat
    \item L’utilisateur demande l’estimation du bénéfice associé au contrat
    \item Le système calcule une estimation du bénéfice, en mettant à jour les indicateurs de performance associés au contrat.
\end{enumerate}
\end{shaded}

\noindent\textsc{\bf{A2 - UC2 :} Consulter le compte rendu des durées des interventions d’un contrat}
\begin{shaded}
\noindent\textsc{Résumé :}\\

L’utilisateur consulte le dossier d’un contrat et demande à obtenir un compte rendu sur les durées des interventions effectuées dans le cadre de ce contrat. \\

\noindent\textsc{Utilisateur :}\\

Responsable d’affaires \\

\noindent\textsc{Scénario :} \\
\begin{enumerate}
    \item L’utilisateur recherche le contrat par sa référence ou le nom de la société dans laquelle il intervient
    \item Le système affiche la liste des contrats pouvant correspondre à la recherche
    \item L’utilisateur sélectionne son contrat
    \item Le système affiche l’état actuel du contrat
    \item L’utilisateur demande des compte rendu sur les durées d’intervention pour ce contrat
    \item Le système lui indique le durée moyenne ainsi que les interventions qui ont demandées le moins ainsi que le plus de temps
\end{enumerate}
\end{shaded}

\noindent\textsc{\bf{A2 - UC3 :} Consulter les statistiques d’un contrat}
\begin{shaded}
\noindent\textsc{Résumé :}\\

L’utilisateur consulte le dossier d’un contrat et demande le récapitulatif des indicateurs de performance relatifs au contrat en cours de consultation. \\

\noindent\textsc{Utilisateur :}\\

Responsable d’affaires ou Commercial \\

\noindent\textsc{Scénario :} \\
\begin{enumerate}
    \item L’utilisateur recherche le contrat par sa référence ou le nom de la société dans laquelle il intervient
    \item Le système affiche la liste des contrats pouvant correspondre à la recherche
    \item L’utilisateur sélectionne son contrat
    \item Le système affiche l’état actuel du contrat
    \item L’utilisateur demande le récapitulatif des indicateurs de performance de ce contrat
    \item Le système actualise les indicateurs et génère un rapport
\end{enumerate}
\end{shaded}

\subsubsection{Services}

\todo{Compléter cette section}

\subsubsection{Identification du bloc applicatif}

\todo{Compléter cette section}

\subsection{Gestion des données}

\subsubsection{Objets métiers}

\todo{Compléter cette section}

\subsubsection{Représentation de la donnée}

\todo{Compléter cette section}

\subsection{Infrastructure}

\subsubsection{Solution retenue}

\todo{Compléter cette section}

\subsubsection{Déploiement de la solution}

\todo{Compléter cette section}

% --------------------------- RISQUE
\section{Analyse des Risques (RISQUE)}% --------------------------- RISQUE
% --------------------------- RISQUE

\subsection{Spécifications fonctionnelles}

\subsubsection{Cas d’utilisations}

\noindent\textsc{\bf{A3 - UC1 :} Calculer l’indice de niveau de sécurité}
\begin{shaded}
\noindent\textsc{Résumé :}\\

Lors de la mise à jour des informations relatives à un contrat ou à une intervention, l’indice de niveau de sécurité est recalculé en intégrant les nouvelles données. \\

\noindent\textsc{Utilisateur :} \\

Commercial ou Responsable d’affaires \\

\noindent\textsc{Scénario :} \\
\begin{enumerate}
    \item L’utilisateur recherche le contrat par sa référence ou le nom de la société dans laquelle il intervient
    \item Le système affiche la liste des contrats pouvant correspondre à la recherche
    \item L’utilisateur sélectionne son contrat
    \item Le système affiche l’état actuel du contrat
    \item L’utilisateur demande le calcul de l’indice de niveau de sécurité du contrat
    \item Le système recalcule l’indice de niveau de sécurité et l’indique à l’utilisateur
\end{enumerate}
\end{shaded}

\noindent\textsc{\bf{A3 - UC2 :} Estimer le niveau de risque d’un contrat}
\begin{shaded}
\noindent\textsc{Résumé :}\\

Lors de la mise à jour des informations relatives à un contrat, le niveau de risque est recalculé en intégrant les nouvelles données. \\

\noindent\textsc{Utilisateur :} \\

Commercial ou Responsable d’affaires \\

\noindent\textsc{Scénario :} \\
\begin{enumerate}
    \item L’utilisateur recherche le contrat par sa référence ou le nom de la société dans laquelle il intervient
    \item Le système affiche la liste des contrats pouvant correspondre à la recherche
    \item L’utilisateur sélectionne son contrat
    \item Le système affiche l’état actuel du contrat
    \item L’utilisateur demande le niveau de risque du contrat
    \item Le système recalcule le niveau de risque et l’indique à l’utilisateur
\end{enumerate}
\end{shaded}

\noindent\textsc{\bf{A3 - UC3 :} Rechercher une solution à un problème référencé}
\begin{shaded}
\noindent\textsc{Résumé :}\\

L’utilisateur souhaite trouver une solution à un problème qu’il suppose référencé dans la base de connaissance. \\

\noindent\textsc{Utilisateur :} \\

Technicien ou Responsable d’affaires \\

\noindent\textsc{Scénario :} \\
\begin{enumerate}
    \item L’utilisateur saisit des mots clés qui décrivent son problème
    \item Une liste de solutions contenant les mots clés est présentée à l’utilisateur
    \item L’utilisateur peut consulter le contenu des différentes solutions afin de trouver une réponse à sa problématique \\
\end{enumerate}
\noindent\textsc{Scénarii alternatifs :}\\
\begin{enumerate}
    \item Aucune solution ne comporte les mots clés saisis et le système ne connaît apparemment pas de réponse à ce problème.
    \item L’utilisateur peut demander la création d’une solution dans la base de connaissance relatif à ce problème et proposer une résolution au problème une fois que celui-ci a été résolu.
\end{enumerate}
\end{shaded}

\subsubsection{Services}

\todo{Compléter cette section}

\subsubsection{Identification du bloc applicatif}

\todo{Compléter cette section}

\subsection{Gestion des données}

\subsubsection{Objets métiers}

\todo{Compléter cette section}

\subsubsection{Représentation de la donnée}

\todo{Compléter cette section}

\subsection{Infrastructure}

\subsubsection{Solution retenue}

\todo{Compléter cette section}

\subsubsection{Déploiement de la solution}

\todo{Compléter cette section}