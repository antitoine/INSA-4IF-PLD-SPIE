% --------------------------- REX
\section{Retour d'Expérience (REX)} % --------------------------- REX
% --------------------------- REX

\subsection{Spécifications fonctionnelles}

\subsubsection{Cas d’utilisations}

% #############################################
\noindent\textsc{\bf{A1 - UC1 :} Consulter une prestation de maintenance}
\begin{shaded}
\noindent\textsc{Résumé :}\\

Pendant une intervention, le technicien souhaite accède à l’historique du dossier pour une prestation de maintenance donné. \\

\noindent\textsc{Utilisateur :}\\

Technicien ou Responsable d’affaires ou Commercial \\

\noindent\textsc{Scénario :} \\
\begin{enumerate}
    \item   L’utilisateur fait une recherche avec la référence d’un contrat ou le nom d’une société
    \item   Le système affiche la liste des contrats pouvant correspondre
    \item   L’utilisateur sélectionne son contrat
    \item   Le système affiche l’état actuel du contrat avec la liste des commandes
    \item   L’utilisateur sélectionne la commande correspondant à son intervention
    \item   Le système affiche l’état actuel de la commande avec la liste des prestations de maintenance
    \item   L’utilisateur sélectionne la prestation de maintenance correspondant à l’intervention
    \item   Le système retourne la prestation de maintenance et les pièces jointes qui lui sont associées \\
\end{enumerate}
\noindent\textsc{Scénarii alternatifs :} \\
\begin{enumerate}
    \item L’utilisateur fait une recherche avec la référence de la commande
    \begin{enumerate}
        \item Le système affiche la liste des commandes pouvant correspondre à la recherche
        \item On reprend le scénario nominal à l’étape de 5 \\
    \end{enumerate}
    \item L’utilisateur fait une recherche avec la référence de la prestation de maintenance
    \begin{enumerate}
        \item Le système affiche la liste des prestations de maintenance pouvant correspondre à la recherche
        \item On reprend le scénario nominal à l’étape 7
    \end{enumerate}
\end{enumerate}
\end{shaded}
% #############################################
\noindent\textsc{\bf{A1 - UC2 :} Saisir le retour d’expérience}
\begin{shaded}
\noindent\textsc{Résumé :}\\

Après une prestation de maintenance effectuée par un technicien, il peut rentrer un résumé et les détails sur cette dernière dans une base de connaissance. \\

\noindent\textsc{Utilisateur :} \\

Technicien \\

\noindent\textsc{Scénario :} \\
\begin{enumerate}
    \item L’utilisateur consulte une prestation de maintenance (A1 - UC1) et demande une saisie de retour d’expérience
    \item Le système demande à l’utilisateur de saisir les informations sur son retour d’expérience
    \item L’utilisateur renseigne les différents champs et valide la saisie
    \item Le système l’enregistre et affiche le résultat de sa saisie \\
\end{enumerate}
\end{shaded}
% #############################################
\noindent\textsc{\bf{A1 - UC3 :} Associer des pièces jointes au retour d’expérience}
\begin{shaded}
\noindent\textsc{Résumé :}\\

L’utilisateur veut associer des pièces jointes de différentes catégories à la prestation de maintenance. \\

\noindent\textsc{Utilisateur :} \\

Technicien \\

\noindent\textsc{Scénario :} \\
\begin{enumerate}
    \item L’utilisateur consulte une prestation de maintenance (A1 - UC1) et demande l’ajout de pièces jointes
    \item L’utilisateur choisit les pièces jointes qu’il veut associer au dossier et leur affecte une catégorie
    \item Les pièces jointes sont récupérées depuis le média qu’il utilise (tablette, ordinateur, etc.)
\end{enumerate}

\end{shaded}

\subsubsection{Services}

\begin{description}
    \item[\textbullet] Rechercher un contrat par référence (identifiant unique) \\
        \it{Description succincte :} ce service permet de récupérer un contrat en utilisant sa référence pour effectuer la recherche. \\
        \it{Complexité :} \bf{simple}.
    \item[\textbullet] Rechercher un contrat par le nom de la société dans lequel il intervient \\
        \it{Description succincte :} ce service permet de récupérer un contrat en utilisant le nom de la société pour effectuer la recherche. \\
        \it{Complexité :} \bf{simple}.
    \item[\textbullet] Récupérer les commandes associées à une référence d’un contrat  \\
        \it{Description succincte :} ce service permet de récupérer les commandes associées à un contrat en utilisant la référence de ce dernier. \\
        \it{Complexité :} \bf{simple}.
    \item[\textbullet] Rechercher une commande par sa référence \\
        \it{Description succincte :} ce service permet de récupérer une commande en utilisant sa référence. \\
        \it{Complexité :} \bf{simple}.
    \item[\textbullet] Récupérer les prestations de maintenance associées à une référence d’une commande \\
        \it{Description succincte :} ce service permet de récupérer les prestations de maintenance en utilisant la référence de la commande associée  \\
        \it{Complexité :} \bf{simple}. 
    \item[\textbullet] Rechercher une prestation de maintenance par sa référence \\
        \it{Description succincte :} ce service permet de récupérer une prestation de maintenance en utilisant sa référence. \\
        \it{Complexité :} \bf{simple}.
    \item[\textbullet] Mise à jour d’une prestation de maintenance \\
        \it{Description succincte :} ce service permet la mise à jour d’une prestation de maintenance en exploitant les saisies de l’utilisateur. \\
        \it{Complexité :} \bf{moyen}.
    \item[\textbullet] Ajouter un document à la prestation de maintenance \\
        \it{Description succincte :} ce service permet de téléverser un document et de l’associer à une prestation de maintenance. \\
        \it{Complexité :} \bf{moyen}.
\end{description}

\subsubsection{Identification du bloc applicatif}

\todo{Compléter cette section}

\subsection{Gestion des données}

\subsubsection{Objets métiers}

\todo{Compléter cette section}

\subsubsection{Représentation de la donnée}

\todo{Compléter cette section}

\subsection{Infrastructure}

\subsubsection{Solution retenue}

\todo{Compléter cette section}

\subsubsection{Déploiement de la solution}

\todo{Compléter cette section}

% --------------------------- PERF
\section{Analyse des Performances (PERF)}% --------------------------- PERF
% --------------------------- PERF

\subsection{Spécifications fonctionnelles}

\subsubsection{Cas d’utilisations}

\noindent\textsc{\bf{A2 - UC1 :} Calculer le bénéfice d’un contrat}
\begin{shaded}
\noindent\textsc{Résumé :}\\

L’utilisateur consulte le dossier d’un contrat puis demande l’estimation du bénéfice associé à celui-ci, tout en estimant les potentiels risques, pouvant survenir lors de l’exécution du contrat. \\

\noindent\textsc{Utilisateur :} \\

Commercial \\

\noindent\textsc{Scénario :} \\
\begin{enumerate}
    \item L’utilisateur recherche le contrat par sa référence ou le nom de la société dans laquelle il intervient
    \item Le système affiche la liste des contrats pouvant correspondre à la recherche
    \item L’utilisateur sélectionne son contrat
    \item Le système affiche l’état actuel du contrat
    \item L’utilisateur demande l’estimation du bénéfice associé au contrat
    \item Le système calcule une estimation du bénéfice, en mettant à jour les indicateurs de performance associés au contrat.
\end{enumerate}
\end{shaded}

\noindent\textsc{\bf{A2 - UC2 :} Consulter le compte rendu des durées des interventions d’un contrat}
\begin{shaded}
\noindent\textsc{Résumé :}\\

L’utilisateur consulte le dossier d’un contrat et demande à obtenir un compte rendu sur les durées des interventions effectuées dans le cadre de ce contrat. \\

\noindent\textsc{Utilisateur :}\\

Responsable d’affaires \\

\noindent\textsc{Scénario :} \\
\begin{enumerate}
    \item L’utilisateur recherche le contrat par sa référence ou le nom de la société dans laquelle il intervient
    \item Le système affiche la liste des contrats pouvant correspondre à la recherche
    \item L’utilisateur sélectionne son contrat
    \item Le système affiche l’état actuel du contrat
    \item L’utilisateur demande des compte rendu sur les durées d’intervention pour ce contrat
    \item Le système lui indique le durée moyenne ainsi que les interventions qui ont demandées le moins ainsi que le plus de temps
\end{enumerate}
\end{shaded}

\noindent\textsc{\bf{A2 - UC3 :} Consulter les statistiques d’un contrat}
\begin{shaded}
\noindent\textsc{Résumé :}\\

L’utilisateur consulte le dossier d’un contrat et demande le récapitulatif des indicateurs de performance relatifs au contrat en cours de consultation. \\

\noindent\textsc{Utilisateur :}\\

Responsable d’affaires ou Commercial \\

\noindent\textsc{Scénario :} \\
\begin{enumerate}
    \item L’utilisateur recherche le contrat par sa référence ou le nom de la société dans laquelle il intervient
    \item Le système affiche la liste des contrats pouvant correspondre à la recherche
    \item L’utilisateur sélectionne son contrat
    \item Le système affiche l’état actuel du contrat
    \item L’utilisateur demande le récapitulatif des indicateurs de performance de ce contrat
    \item Le système actualise les indicateurs et génère un rapport
\end{enumerate}
\end{shaded}

\subsubsection{Services}

\begin{description}
    \item[\textbullet] Rechercher un contrat par référence (identifiant unique) \\
        \it{Description succincte :} ce service permet de récupérer un contrat en utilisant sa référence pour la recherche. \\
        \it{Complexité :} \bf{simple}.
    \item[\textbullet] Rechercher un contrat par le nom de la société dans lequel il intervient \\
        \it{Description succincte :} ce service permet de récupérer un contrat en utilisant le nom de la société pour la recherche. \\
        \it{Complexité :} \bf{simple}.
    \item[\textbullet] Récupérer les commandes associées à une référence d’un contrat  \\
        \it{Description succincte :} ce service permet de récupérer les commandes en utilisant la référence du contrat qui leur est associé. \\
        \it{Complexité :} \bf{simple}.
    \item[\textbullet] Récupérer les prestations de maintenance associées à une référence d’une commande \\
        \it{Description succincte :} ce service permet de récupérer les prestations de maintenance en utilisant la référence de la commande associée.  \\
        \it{Complexité :} \bf{simple}.
    \item[\textbullet] Calculer l’estimation du bénéfice du contrat \\
        \it{Description succincte :} ce service permet de calculer l’estimation du bénéfice d’un contrat à partir des données du SI. \\
        \it{Complexité :} \bf{moyen}.
    \item[\textbullet] Actualiser les indicateurs de performance d’un contrat \\
        \it{Description succincte :} ce service permet d’actualiser les indicateurs du contrat à partir des informations contenues dans le SI. \\
        \it{Complexité :} \bf{moyen} à \bf{complexe}.
    \item[\textbullet] Générer un rapport sur les indicateurs du contrat \\
        \it{Description succincte :} ce service permet de générer un rapport de synthèse des indicateurs. \\
        \it{Complexité :} \bf{moyen} à \bf{complexe}.
    \item[\textbullet] Calculer la durée moyenne de toutes les prestations de maintenance d’un contrat \\
        \it{Description succincte :} ce service permet de calculer, à partir des données du SI, la durée moyenne des prestations de maintenance. \\
        \it{Complexité :} \bf{moyen}.
    \item[\textbullet] Récupérer les prestations de maintenance qui ont demandées le plus de temps \\
        \it{Description succincte :} ce service permet de récupérer les prestations de maintenance qui ont durée le plus longtemps en utilisant un seuil minimal. \\
        \it{Complexité :} \bf{moyen}.
    \item[\textbullet] Récupérer les prestations de maintenance qui ont demandées le moins de temps \\
        \it{Description succincte :} ce service permet de récupérer les prestations de maintenance qui ont durée le moins longtemps en utilisant un seuil maximal. \\
        \it{Complexité :} \bf{moyen}.
\end{description}

\subsubsection{Identification du bloc applicatif}

\todo{Compléter cette section}

\subsection{Gestion des données}

\subsubsection{Objets métiers}

\todo{Compléter cette section}

\subsubsection{Représentation de la donnée}

\todo{Compléter cette section}

\subsection{Infrastructure}

\subsubsection{Solution retenue}

\todo{Compléter cette section}

\subsubsection{Déploiement de la solution}

\todo{Compléter cette section}

% --------------------------- RISQUE
\section{Analyse des Risques (RISQUE)}% --------------------------- RISQUE
% --------------------------- RISQUE

\subsection{Spécifications fonctionnelles}

\subsubsection{Cas d’utilisations}

\noindent\textsc{\bf{A3 - UC1 :} Calculer l’indice de niveau de sécurité}
\begin{shaded}
\noindent\textsc{Résumé :}\\

Lors de la mise à jour des informations relatives à un contrat ou à une intervention, l’indice de niveau de sécurité est recalculé en intégrant les nouvelles données. \\

\noindent\textsc{Utilisateur :} \\

Commercial ou Responsable d’affaires \\

\noindent\textsc{Scénario :} \\
\begin{enumerate}
    \item L’utilisateur recherche le contrat par sa référence ou le nom de la société dans laquelle il intervient
    \item Le système affiche la liste des contrats pouvant correspondre à la recherche
    \item L’utilisateur sélectionne son contrat
    \item Le système affiche l’état actuel du contrat
    \item L’utilisateur demande le calcul de l’indice de niveau de sécurité du contrat
    \item Le système recalcule l’indice de niveau de sécurité et l’indique à l’utilisateur
\end{enumerate}
\end{shaded}

\noindent\textsc{\bf{A3 - UC2 :} Estimer le niveau de risque d’un contrat}
\begin{shaded}
\noindent\textsc{Résumé :}\\

Lors de la mise à jour des informations relatives à un contrat, le niveau de risque est recalculé en intégrant les nouvelles données. \\

\noindent\textsc{Utilisateur :} \\

Commercial ou Responsable d’affaires \\

\noindent\textsc{Scénario :} \\
\begin{enumerate}
    \item L’utilisateur recherche le contrat par sa référence ou le nom de la société dans laquelle il intervient
    \item Le système affiche la liste des contrats pouvant correspondre à la recherche
    \item L’utilisateur sélectionne son contrat
    \item Le système affiche l’état actuel du contrat
    \item L’utilisateur demande le niveau de risque du contrat
    \item Le système recalcule le niveau de risque et l’indique à l’utilisateur
\end{enumerate}
\end{shaded}

\noindent\textsc{\bf{A3 - UC3 :} Rechercher une solution à un problème référencé}
\begin{shaded}
\noindent\textsc{Résumé :}\\

L’utilisateur souhaite trouver une solution à un problème qu’il suppose référencé dans la base de connaissance. \\

\noindent\textsc{Utilisateur :} \\

Technicien ou Responsable d’affaires \\

\noindent\textsc{Scénario :} \\
\begin{enumerate}
    \item L’utilisateur saisit des mots clés qui décrivent son problème
    \item Une liste de solutions contenant les mots clés est présentée à l’utilisateur
    \item L’utilisateur peut consulter le contenu des différentes solutions afin de trouver une réponse à sa problématique \\
\end{enumerate}
\noindent\textsc{Scénarii alternatifs :}\\
\begin{enumerate}
    \item Aucune solution ne comporte les mots clés saisis et le système ne connaît apparemment pas de réponse à ce problème.
    \item L’utilisateur peut demander la création d’une solution dans la base de connaissance relatif à ce problème et proposer une résolution au problème une fois que celui-ci a été résolu.
\end{enumerate}
\end{shaded}

\subsubsection{Services}

\begin{description}
    \item[\textbullet] Rechercher un contrat par référence (identifiant unique) \\
        \it{Description succincte :} ce service permet de récupérer un contrat en utilisant sa référence pour la recherche. \\
        \it{Complexité :} \bf{simple}.
    \item[\textbullet] Rechercher un contrat par le nom de la société dans lequel il intervient \\
        \it{Description succincte :} ce service permet de de récupérer un contrat de maintenance en utilisant le nom de la société pour la recherche. \\
        \it{Complexité :} \bf{simple}.
    \item[\textbullet] Calculer l’indice de niveau de sécurité du contrat \\
        \it{Description succincte :} ce service permet de calculer l’indice de niveau de sécurité à partir des informations présentes dans le SI. \\
        \it{Complexité :} \bf{complexe}.
    \item[\textbullet] Calculer l’indice de risque du contrat \\
        \it{Description succincte :} ce service permet de calculer l’indice de risque à partir des informations présentes dans le SI. \\
        \it{Complexité :} \bf{complexe}.
    \item[\textbullet] Rechercher une solution par des mots-clés \\
        \it{Description succincte :} ce service permet de récupérer des solutions contenant des mots-clés ou liées à des concepts présent dans la recherche. \\
        \it{Complexité :} \bf{moyen} à \bf{complexe}.
    \item[\textbullet] Rechercher une solution par son identifiant \\
        \it{Description succincte :} ce service permet de récupérer une solution en utilisant son identifiant pour effectuer la rechercher. \\
        \it{Complexité :} \bf{simple}.
    \item[\textbullet] Créer une nouvelle solution à un problème \\
        \it{Description succincte :} ce service permet d’éditer une nouvelle solution à un problème répertorié en exploitant les saisies de l’utilisateur. \\
        \it{Complexité :} \bf{moyen}.
    \item[\textbullet] Modifier une solution présente dans la base de connaissance \\
        \it{Description succincte :} ce service permet de modifier une solution précédemment ajoutée \\
        \it{Complexité :} \bf{moyen}.
\end{description}

\subsubsection{Identification du bloc applicatif}

\todo{Compléter cette section}

\subsection{Gestion des données}

\subsubsection{Objets métiers}

\todo{Compléter cette section}

\subsubsection{Représentation de la donnée}

\todo{Compléter cette section}

\subsection{Infrastructure}

\subsubsection{Solution retenue}

\todo{Compléter cette section}

\subsubsection{Déploiement de la solution}

\todo{Compléter cette section}