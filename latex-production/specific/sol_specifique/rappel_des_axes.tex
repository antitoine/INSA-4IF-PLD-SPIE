\section{Retour d'Expérience}

Afin d’améliorer l’efficacité de ses processus et l’accès au reporting, il peut être intéressant pour SPIE de doter ses collaborateurs d’appareils mobiles ainsi que d’applications dédiées à ses besoins. En effet, la mise en place d’une application mobile pour les techniciens réalisant les différentes tâches de maintenance pourra favoriser la rédaction de retours d’expériences, directement à partir de leur terminal mobile. L’application, communiquant avec le système d’information de SPIE Sud-Est, permettra ainsi d’alimenter la base de connaissances de SPIE et permettant, à termes, d’exploiter les retours d’expérience afin d’améliorer la qualité des services fournis par SPIE. \\

L’application mobile mise en place pourra être modulaire, permettant ainsi de développer progressivement les modules correspondant aux besoins des différents acteurs de la société. Il faut également noter que l’utilisation d’une application mobile par les acteurs d’une société permet de renforcer l’image de celle-ci.

\section{Analyse des Performances}

L’amélioration des performances passe par la mise en place d’indicateurs de performance. Ces indicateurs de performance seront accessibles depuis un tableau de bord qui permettra à terme aux employés de SPIE de suivre les évolutions des activités menées. Ce type de tableau de bord réalise la synthèse des indicateurs pour SPIE ce qui facilitera également la prise de décisions concernant les orientations à donner aux projets. \\

Cet axe requiert également la standardisation des mesures réalisées pour calculer les indicateurs exposés sur le tableau de bord.

\section{Analyse des Risques}

Actuellement, l’analyse des risques est effectuée ponctuellement, sans réel suivi sur le long terme. Il serait donc intéressant de mettre en oeuvre une gestion des risques plus poussée et plus régulière, qui interviendrait tout au long de la réalisation du contrat. \\

Cette amélioration s’inscrit dans une optique de Business Intelligence, ou BI, en combinaison avec une base de connaissance. En effet, l’accumulation des données concernant la gestion des risques, grâce à une nouvelle surveillance accrue, permettra de tirer parti des différentes informations accumulées au fil des expériences.