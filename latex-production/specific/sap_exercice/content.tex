\section{Prise en main}

L'application SAP ByD semble très dense au premier lancement et assez peu organisée globalement. Cette première impression est sans doute due au grand nombre de fonctions qu’elle propose.\\

La vidéo \verb ByD_Basics_1308_FR.mp4.mpeg  permet de faire un tour rapide du logiciel. On peut notamment voir que presque toutes les vues du logiciel sont personnalisables, et que la quantité de paramètres est énorme. \\

Connaître les différents endroits où l'on peut faire des recherches (entreprise search et recherche par poste de travail) permet de gagner du temps dans la navigation. \\

La bibliothèque de SAP n’est pas des plus pratique à utiliser car les sous menus sont nombreux et répétitifs. Le nombre de documents à lire est également important, et il est impossible de comprendre l'enjeu de certains scénarii sans les avoir étudiés en profondeur ou vu comment ceux-ci se déroulent sur le terrain. 

\section{Comment la plateforme SAP peut être correctement utilisée par notre équipe ?}

La plateforme SAP nous permet de nous familiariser avec le type d'outil que l'on peut retrouver dans les grandes entreprises aujourd'hui, c'est à dire des outils aux fonctionnalités nombreuses, regroupant un une grande quantité de scénarii métiers. \\

Ensuite, cela nous permet de faire une comparaison entre les choix mis en place par les concepteurs de SAP au niveau des scéanrii métiers et ceux de SPIE. En effet, il peut être intéressant d'avoir plusieurs points de vue concernant la réalisation d'un même scénario. Cela nous permet également de pouvoir proposer à SPIE de nouvelles solutions pour améliorer leur scénario existant en reprenant certaines des idées utilisées dans SAP. \\

Enfin, il s'agit d'un outil qui pourrait être proposé à SPIE pour gérer certains métiers de SPIE. SAP pourrait, en effet, être un outil de choix pour remplacer, complémenter peopleSoft là où il n'est pas très adapté. \\

SAP peut ainsi être utilisé en complément de PeopleSoft. Cela permet une transition “douce” d'un ERP à un autre, en intégrant lentement chacun des processus de PeopleSoft vers SAP. Ainsi, malgré l'utilisation de deux ERP en simultané, on diminue les coûts car la formation des employés se fait petit à petit. On peut également choisir de garder uniquement ce qui se fait de mieux chez les différents ERP, pour pouvoir toucher un marché plus important, ou de manière à ce que SPIE puisse facilement élargir son offre par la suite.

\section{Retour sur l’exercice "Ordre de service"}

\subsection{Ce que nous avons compris}

Cette rubrique de gestion permet de gérer les ordres de service, c'est-à-dire les documents de base utilisés dans la gestion de toutes les activités de service et de réparation en clientèle. \\

Il est ainsi possible planifier les services à fournir, notamment la gestion de toutes les ressources, comme les pièces de rechange et la main-d'\oe{}uvre, ainsi que l'affectation des droits appropriés. \\ 

La rubrique de gestion "Ordre de service" enregistre les coûts et les produits de la prestation de service à des fins de contrôle des marges de profit. Elle établit les historiques des réparations et facilite la capture des connaissances pour l'établissement.

Etant donné que l'on considère principalement tout ce qui concerne la maintenance chez SPIE, il est essentiel de comprendre comment fonctionne ce scénario.

\subsection{Les problèmes que nous avons rencontrés}

La plateforme SAP présente à première vue une interface assez founie mais bien organisée. La vidéo de formation est très complémentaire de cette organisation et permet d'appréhender plus facilement l'interface qu'au premier abord. \\

Gloabalement, la plateforme se révèle accesible et nous n'avons relevé aucun problème bloquant ou insoluble.

\section{Bilan}

La vidéo et des conseils seront fournis au fur et à mesure que les questions sur la plateforme sont soumises. Il n'est donc pas envisagé à ce jour d'apporter une formation complémentaire sur cet outil. \\
Enfin, tous les membres de l'équipe s'étant prétés à l'exercice et ayant exprimés leur bonne compréhension du sujet le bilan est globalement positif.

