\begin{shaded}
    \textsc{Sources} :
    \begin{itemize}
        \item[\textbullet] http://www.spie.com/sites/default/files/spie\_document\_de\_base.pdf (page 57)
        \item[\textbullet] wikipédia SPIE
        \item[\textbullet] wikipedia des différents groupes cités
    \end{itemize}
\end{shaded}

\section{Le groupe SPIE}

Le groupe SPIE est le leader européen indépendant des services multi-techniques. \\

La principale activité du groupe consiste à fournir des services multi-techniques dans trois zones géographiques : la France, l’Allemagne \& Europe Centrale et l’Europe du Nord-Ouest. Il fournit également des services d’intégrations et de support à l’exploitation des systèmes de communications et d’informations essentiellement en France. Par ailleurs, il propose  dans le cadre de ses activités Pétrole-Gaz et Nucléaire des services et son expertise technique dans les secteurs spécialisés de l’industrie pétrolière. \\

Il est l’un des premiers acteurs indépendants en France, dans un marché relativement consolidé où les grands acteurs nationaux occupent une place prépondérante mais où il existe encore un nombre important d’acteurs locaux. \\ 

En outre, le Groupe bénéficie d’une présence solide et croissante en Allemagne, aux Pays-Bas, en Belgique et au Royaume-Uni, marchés sur lesquels il considère être un des principaux acteurs.

\section{En france}

\subsection{Description générale}

Le marché français des services multi-techniques est structuré autour de quatre types d’acteurs : 
Les grandes filiales des principaux groupes français de bâtiment-travaux publics (Vinci Energies, Eiffage Energie, Bouygues E\&S). 
Les filiales de groupes de fournisseurs d’énergie (Cofely – GDF Suez)
Les grands acteurs nationaux indépendants (SPIE, SNEF), principalement axés sur les services mécaniques et électriques.
Diverses entreprises de petite et moyenne taille, locales et régionales, dont la stratégie est basée sur la proximité et la relation avec les clients. \\

Les cinq acteurs principaux représentent actuellement environ 39 \% du marché. Les principaux acteurs offrent aujourd’hui tous types de services et couvrent tous les secteurs d’activité. En 2013, sur un marché français toujours fragmenté, bien que plus consolidé que les autres marchés européens, SPIE estime être le troisième acteur avec une part de marché d’environ 7\%.

\subsection{Descriptions de quelques concurrents}
 
\subsubsection{VINCI énergies}
\hrule

\begin{shaded}
    \noindent\textsc{Caractéristiques} : \\
    \begin{itemize}
        \item CA : 9Mds €
        \item Effectif : 64 000 employés
        \item Présence : mondiale dans 45 pays au sein de 1500 entreprises
        \item ERP : Anciennement Oracle e-Business, SAP Hana depuis 2012\\
    \end{itemize}
\end{shaded}

\noindent\textsc{Description} : \\

Filiale du groupe VINCI spécialisée dans l’énergie. Pour son activité de maintenance, elle cible les secteurs industriels et tertiaires. Elle rentre en concurrence avec SPIE sur les aspects de maintenance énergétiques industrielle.

\subsubsection{CEGELEC}
\hrule

\begin{shaded}
    \noindent\textsc{Caractéristiques} : \\
    \begin{itemize}
        \item CA : 3 Mds €
        \item Effectif : 26 700 employés
        \item Présence : dans 30 pays
        \item ERP : SAP
    \end{itemize}
\end{shaded}

\noindent\textsc{Description} : \\

Filiale du groupe VINCI depuis 2010, elle est spécialisée dans les activités suivantes : Énergie et électricité, Automatisation, instrumentation et contrôle, Technologies d’information et de communication, Génie climatique, mécanique, Maintenance et services. \\

Lors de ses activités de maintenance, elle priorise une relation directe entre le technicien de maintenance et le client afin d’assurer une meilleure qualité de l’activité. Les équipes de maintenance sont totalement dédiées à leur client. C’est à dire qu’une équipe ne travaille que pour un client à la fois.

\subsubsection{Bouygues Construction}
\hrule

\begin{shaded}
    \noindent\textsc{Caractéristiques} : \\
    \begin{itemize}
        \item CA : 10 401 Mds€ (Bouygues Construction) (juillet 2013)
        \item Effectif : 55 381 collaborateurs (Bouygues Construction) (juillet 2012)
        \item ERP : Qualiac RIA Web 2.0
    \end{itemize}
\end{shaded}

\noindent\textsc{Description} : \\

Filiale du groupe Bouygues, Bouygues Construction a pour domaine de compétences le BTP ainsi que des activités d’énergie et de maintenance. Parmi les réalisations de la filiale, on retrouve des constructions durables (viaducs, etc.), aussi bien en France qu’à l’étranger, ainsi que des réalisations immobilières et/ou du service public

\subsubsection{Eiffage Energie}
\hrule

\begin{shaded}
    \noindent\textsc{Caractéristiques} : \\
    \begin{itemize}
        \item CA : 3,3 Md€ en 2014
        \item Effectif : 25 000 Collaborateurs 
        \item ERP : aucune information, ERP interne il semblerait
    \end{itemize}
\end{shaded}
    
\noindent\textsc{Description} : \\

Eiffage Énergie est la branche énergie du groupe Eiffage, troisième groupe de BTP et concessions français. \\

Eiffage Énergie conçoit, réalise et exploite des réseaux et systèmes d'énergie et d'information, à destination des collectivités, de l’industrie et du tertiaire.
Spécialisé dans le génie électrique, le génie climatique, le génie mécanique et les télécommunications, de la conception à l’exploitation-maintenance, la branche Énergie du groupe Eiffage compte plus de 25 000 collaborateurs et a réalisé un chiffre d’affaires de 3,3 milliards d’euros en 2014.

\section{Dans le reste de l'Europe}
    
L’analyse de l’ensemble des concurrents de SPIE dans le reste du monde et de l’Europe étant extrêmement vaste, nous allons nous concentrer sur l’Allemagne et le Royaume-Uni qui sont les régions où le marché de SPIE est important.

\subsection{En Allemagne}

Le marché allemand des services multi-techniques a généré un chiffre d’affaires total d’environ 71 milliards d’euros en 2013. Avec l’acquisition des activités Service Solutions d’Hochtief en 2013, l’Allemagne est aujourd’hui le deuxième marché du Groupe.  \\

\noindent Le marché allemand est structuré autour de cinq types d’acteurs : 
\begin{itemize}
    \item Les fournisseurs de solutions techniques pour des grands projets d’installation ou de rénovation (Caverion, Imtech, Cofely); 
    \item les acteurs de facilities management technique (SPIE, Bilfinger, Strabag, Vinci);
    \item les prestataires de services industriels spécialisés (Voith Services, Currenta, Bilfinger);
    \item les acteurs centrés sur le facilities management non technique (nettoyage, restauration) (Sodexo, Wisag, Compass, Dussman) ;
    \item divers acteurs locaux. 
\end{itemize}

\subsection{Au Royaume-Uni}

Le marché britannique des services multi-techniques a généré un chiffre d’affaires total d’environ 21 milliards d’euros en 2013. \\

\noindent Le marché britannique est structuré autour de quatre types d’acteurs : 
\begin{itemize}
    \item Les groupes de construction intégrés (Balfour Beatty, Skanska Rashleigh Weatherfoil, Crown House) ;
    \item les groupes spécialisés dans le secteur des services multi-techniques (NG Bailey, SPIE, Forth Electrical, Imtech, T. Clarke) ;
    \item les opérateurs présents sur d’autres secteurs de services mais proposant une offre de services en ingénierie mécanique et électrique (SSE contracting, Lorne Stewart) ;
    \item diverses entreprises de moyenne et petite taille, locales et régionales.
\end{itemize}

\section{Conclusion}

En comparant les concurrents de SPIE avec SPIE, nous remarquons que SPIE a un chiffre
d’affaires beaucoup moins élevé que ses concurrents. Cela s’explique essentiellement par le fait que SPIE s’est spécialisé dans des domaines plus spécifiques comme les énergies, les collectivités
