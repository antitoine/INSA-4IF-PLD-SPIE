\section{Contraintes générales}

Afin qu’un document puisse être considéré comme conforme puis, in fine, valide, il est nécessaire de définir, pour chaque livrable, les différentes contraintes à respecter. Ces contraintes s’appliquant à la fois sur la structure et le contenu des livrables. Nous pouvons définir pour ces derniers leur squelette, à travers la définition de leur plan.

\section{Squelettes documentaires}

\bf{Livrable DI/4401/1 - Dossier d’initialisation} \\

\begin{enumerate}
    \item Objet et contexte du projet
    \item Livrables
    \item Méthode et outils choisis
    \item Identification des activités et des tâches
    \item Planning
    \item Organisation de l’équipe
    \item Procédures validation/recette
    \item Gestion des risques \\
\end{enumerate}

\bf{Livrable PAQ/4401/2 - Dossier qualité} \\

\begin{enumerate}
    \item Objectifs
    \item Domaine d'application
    \item Gestion de la documentation
    \item Cycle de vie des documents
    \item Contrainte de validation 
    \item Outils utilisés
    \item Conclusion \\
\end{enumerate}

\bf{Livrable EE/4401/3 - Étude de l’existant} \\

\begin{enumerate}
    \item Contexte de l'étude
    \item Périmètre métier et fonctionnel
    \item Description du système d'information organisé
        \begin{enumerate}
            \item Les processus métier/fonctionnels
            \item La description des données
            \item La structure organisationnelle
        \end{enumerate}
    \item Description du système informatique
        \begin{enumerate}
            \item Les applications
            \item L'architecture technique \\
        \end{enumerate}
\end{enumerate}

\bf{Livrable BM/4401/4 - Benchmarking} \\

\begin{enumerate}
    \item Synthèse de scénarios de processus
    \item Progiciels généralistes ou spécialisés et principales fonctionnalités
    \item Analyse de l'adéquation globale des solutions et référentiels présentés \\
\end{enumerate}

\bf{Livrable SSC/4401/5 - Spécification du SI cible} \\

\begin{enumerate}
    \item Nouveaux modèles
    \item Modèles de l'existant modifiés
    \item Règles de gestion principales
    \item Axes d'amélioration \\
\end{enumerate}

\bf{Livrable SSP/4401/6 - Spécification d’une solution spécifique} \\

\begin{enumerate}
    \item Solution Informatique
        \begin{enumerate}
            \item Architecture applicative
                \begin{enumerate}
                    \item Liste des blocs applicatifs
                    \item Description des blocs : outils/service et données
                    \item Échange de données entre les blocs
                    \item Modèle d’architecture d’exécution
                    \item Schéma général
                \end{enumerate}
            \item Architecture technique
                \begin{enumerate}
                    \item Éléments actifs : réseau, serveurs, postes de travail
                    \item Architecture logicielle
                \end{enumerate}
        \end{enumerate}
    \item Solution organisationnelle \\
\end{enumerate}

\bf{Livrable SST/4401/7 - Spécification d’une solution standard} \\

\begin{enumerate}
    \item Introduction
    \item Vue globale 
    \item Vue organisationnelle 
    \item Vue informationnelle macro 
    \item Vue fonctionnelle
    \item Glossaire \\
\end{enumerate}

\bf{Livrable MCE/4401/8 - Modélisation et configuration d’une solution ERP} \\

\begin{enumerate}
    \item Introduction
    \item Vue globale de la chaîne de valeur SPIE
    \item Vue organisationnelle 
    \item Vue informationnelle 
    \item Vue fonctionnelle
    \item Glossaire  \\
\end{enumerate}

\bf{Livrable DC/4401/9 - Dossier de choix} \\

\begin{enumerate}
    \item Pour chaque solution
        \begin{enumerate}
            \item Rappel des fonctionnalités de la solution informatique et organisationnelle
            \item Chiffrages des coûts
            \item Retour sur investissement
            \item Évaluation des autres critères de comparaison
        \end{enumerate}
    \item Tableau comparatif et pondération
    \item Recommandations \\
\end{enumerate}
 
\bf{Livrable RDB/4401/10 - Restitution document bilan} \\

\begin{enumerate}
    \item Résumé des documents précédents
    \item Évolution du produit attendu
    \item Bilan des charges \\
\end{enumerate}

\bf{Livrable DB/4401/11 - Dossier bilan} \\

\begin{enumerate}
    \item Les phases du projet
        \begin{enumerate}
            \item Phase d’organisation du projet
            \item Phase d’expression des besoins
            \item Phase d’analyse de scénario
            \item Phase d’évaluation du projet
        \end{enumerate}
    \item Répartition de la charge de travail
    \item Suivi du moral et de la bonne humeur
    \item Suivi de la qualité
    \item Bilans personnels 
\end{enumerate}

\section{Identification des documents}

Les documents constituant les livrables sont tous identifiés par  une référence unique qui est construite de la manière suivante \it{$<$CodeTextuelDuLivrable$>$}/4401/\it{$<$NumeroDeVersionDuLivrable$>$}.

Le \it{$<$CodeTextuelDuLivrable$>$} est construit sur l'intitulé du livrable, généralement les initiales des mots constituant cet intitulé. 

Le \it{$<$NumeroDeVersionDuLivrable$>$} est quant à lui utilisé pour réaliser l'historique des révisions et permet au client de suivre les corrections apportées.
