
\section{Généralités}

La recherche et l’élaboration d’une solution pour l’entreprise vise à livrer un ensemble de livrable spécifiant le résultat proposé. C’est dans le cadre de la réalisation de ces livrables qu’intervient le Plan d'Assurance Qualité (PAQ) afin de certifier l’état des documents produits. Ce document permet notamment de décrire les processus pour créer, sauvegarder, archiver et envoyer un livrable.

\section{Évolution du PAQ}

Le plan d’assurance qualité a été défini en début de projet et ne peut pas faire l’énumération ou l’explication de toutes les démarches de qualité. C’est pourquoi ce document est amené à évoluer pendant le déroulement du projet notamment pour les cas suivants : \\

\begin{itemize}
    \item[\textbullet] Défaut, incohérence ou imprécision dans le PAQ.
    \item[\textbullet] Après réflexion, une meilleure solution a été trouvée et devrait être appliquée.
    \item[\textbullet] Une bonne pratique a été identifié, il faut la formaliser et l’ajouter. \\
\end{itemize}

Tout changement doit avoir l’accord du responsable qualité et du chef de projet avant son intégration au PAQ. \\

Chaque modification du PAQ devra être notifiée à l’ensemble de l’équipe projet et chaque membre devra en prendre connaissance et s’informer en cas de questionnement ou d’incompréhension.
    
\section{En cas de non-respect du PAQ}
    
Dans les cas de non-respect du PAQ, la personne en charge du document à livrer sera directement informé par le responsable qualité ou le chef de projet afin de corriger les défauts. En aucun cas le document sera validé si le plan qualité n’est pas entièrement appliqué. \\

Un non-respect du plan qualité peut avoir de grandes conséquences. En effet, un retard plus ou moins important en découlera, avec une livraison dans les temps plus complexe et risqué et éventuellement un dépassement des temps prévus et donc des coûts. C’est pourquoi, les membres de l’équipe doivent bien être conscients des risques afin de prendre à la lettre les règles du PAQ.
    
\section{Procédure de dérogation}

La rigueur du PAQ est là pour garantir une harmonie des livrables et un bon déroulement de projet. Cependant dans certains cas très précis et justifiés, une dérogation pourra être délivré suite à l’accord du chef de projet et du responsable qualité. Si ce n’est pas le cas, la procédure de non-respect du PAQ sera appliquée.
