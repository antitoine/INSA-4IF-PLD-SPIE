\section{Gestion de projet}

Pour communiquer facilement dans l’équipe, nous utilisons un outil spécialisé dans le domaine, Slack. En effet, ce dernier permet de communiquer sur le projet, quelque soit la localisation des membres de l’équipe et quelque soit l’équipement technique de ces derniers. \\

Pour gérer l’avancement des différentes tâches et avoir une vision des responsabilités de chacun, nous utilisons l’outil Google Sheet, assimilable à un classeur Excel, qui permet également d’avoir des statistiques sur les temps passés, les avancements et les délais.

\section{Gestion des documents}

Comme indiqué précédemment, nous utilisons dans un premier temps un outil dédié au travail collaboratif pour éditer les documents, Google Docs. Puis, afin d’appliquer une mise en page identique à tous les documents, nous utilisons l’outil LaTeX à travers l’éditeur TeXMeX.

\section{Gestion des livrables}

Afin de gérer les différentes versions des livrables et les stocker facilement, nous utilisons à la fois Google Drive ainsi que l’outil spécialisé Git à travers un dépôt hébergé sur un serveur personnel.

\section{Gestion des diagrammes}

La gestion des diagrammes est réalisée à travers divers outils, en fonction du type de digramme à gérer. Concernant les diagrammes répondant de la méthode UML, l’outil PlantUML est utilisé, permettant ainsi d’uniformiser les différents diagrammes, grâce à une description textuelle de ces derniers. \\

Les digrammes ARIS sont réalisés grâce à l’outil spécialisé ARIS Architect, qui offre l’ensemble des méthodes et composants nécessaires à la réalisation des différents modèles.

\section{Gestion de données ERP}

Concernant la gestion des informations reliées aux ERP, l’outil spécialisé SAP est utilisé, mettant ainsi à disposition toutes les fonctionnalités nécessaires au projet, tout en profitant de l’expertise du responsable SAP de l’équipe projet.